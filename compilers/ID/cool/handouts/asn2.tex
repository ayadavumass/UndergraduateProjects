\documentclass[11pt]{article}
\usepackage{handout}
%
% Copyright (c) 1995-1996 by Alex Aiken.  All rights reserved.
% Permission is granted to modify and distribute this document for
% for non-commercial purposes, so long as this copyright notice is retained
% in all copies.
%
% Side margins:
% Actual margin is 1 in + this number
\oddsidemargin -0.25in
\evensidemargin -0.25in

% Text width:
\textwidth 6.9in

% Top margin:
% Actual margin is 1.5 in + this number
\topmargin -.3in

% Text height:
\textheight 8.7in

% Handout Numbers

\newcommand{\PAOneNum}{1}
\newcommand{\WAOneNum}{2}
\newcommand{\PATwoNum}{3}
\newcommand{\PAThreeNum}{4}
\newcommand{\WATwoNum}{5}
\newcommand{\WATwoSolNum}{6}
\newcommand{\WAOneSolNum}{7}
\newcommand{\WAThreeNum}{8}
\newcommand{\WAFourNum}{9}
\newcommand{\WAThreeSolNum}{10}
\newcommand{\PAFourNum}{11}
\newcommand{\WAFourSolNum}{12}
\newcommand{\WAFiveNum}{13}
\newcommand{\WAFiveSolNum}{14}
\newcommand{\WASixNum}{15}
\newcommand{\WASixSolNum}{16}
\newcommand{\PAFiveNum}{17}
\newcommand{\WASevenNum}{18}
\newcommand{\WASevenSolNum}{19}
\newcommand{\PAExtraNum}{20}
\newcommand{\WAEightNum}{21}
\newcommand{\WAEightSolNum}{22}
\newcommand{\PASixNum}{23}
\newcommand{\WANineNum}{24}
\newcommand{\WANineSolNum}{25}
\newcommand{\WATenNum}{26}
\newcommand{\WATenSolNum}{27}
\newcommand{\WAElevenNum}{28}
\newcommand{\WAElevenSolNum}{29}

%% generally useful macros for writing
\newcommand{\TexDir}{/home3/aiken/tex}

% These allow switching interline spacing; the change takes effect immediately:

\makeatletter
\newcommand{\singlespacing}{\let\CS=\@currsize\renewcommand{\baselinestretch}{1}\tiny\CS}
\newcommand{\oneandahalfspacing}{\let\CS=\@currsize\renewcommand{\baselinestretch}{1.25}\tiny\CS}
\newcommand{\doublespacing}{\let\CS=\@currsize\renewcommand{\baselinestretch}{1.5}\tiny\CS}
%setspacingto sets the interline spacing to the value of its argument
% e.g., \setspacingto{1.5} is the same as \doublespacing
\newcommand{\setspacingto}[1]{\let\CS=\@currsize\renewcommand{\baselinestretch}{#1}\tiny\CS}
\makeatother

% nonumber
\newcommand{\nn}{\nonumber}

% Tab for hand-formatting:
\newcommand{\tab}{\hspace*{2em}}

% Angle brackets:
\newcommand{\la}{\langle}
\newcommand{\ra}{\rangle}

% s.t.
\newcommand{\st}{\mbox{\ s.t.\ }}

% otherwise
\newcommand{\ow}{\m{\rm otherwise}}

% if
\newcommand{\mif}{\m{\rm if\ }}

% Harpoons
\newcommand{\rh}{\rightharpoonup}
\newcommand{\lh}{\leftharpoonup}

% Denotational-semantics-style brackets and bottom:
\newcommand{\lbk}{\lbrack\!\lbrack}
\newcommand{\rbk}{\rbrack\!\rbrack}
\newcommand{\bottom}{\perp}

% Projection operator
\newcommand{\proj}{\!\downarrow\!}

% macros for mbox combined with another style
% (useful for changing typefaces in math mode)
\newcommand {\mboxbf}[1]{\mbox{{\bf #1}}}
\newcommand {\mboxit}[1]{\mbox{{\it #1}}}
\newcommand {\mboxem}[1]{\mbox{{\em #1}}}
\newcommand {\m}{\mbox}
\newcommand {\ch}{\rm}
	
% macro for creating a binary operator
%
% example:  \makebinop{\makebinop{\mybmod}{mod}
%	(duplicates the \bmod macro)
%
\def\makebinop#1#2{\def#1{\mskip-\medmuskip \mskip5mu
\mathbin{\rm #2} \penalty900 \mskip5mu \mskip-\medmuskip}}

% Environments for theorems, lemmas, etc.
\newtheorem{theorem}{Theorem}[section]
\newtheorem{lemma}[theorem]{Lemma}
\newtheorem{corollary}[theorem]{Corollary}
\newtheorem{definition}[theorem]{Definition}
\newtheorem{observation}[theorem]{Observation}
\newtheorem{fact}{Fact}[theorem]
\newtheorem{proposition}[theorem]{Proposition}
\newtheorem{example}[theorem]{Example}
\newtheorem{constraint}[theorem]{Constraint}
\newtheorem{axiom}[theorem]{Axiom}
\newtheorem{law}[theorem]{Law}
\newtheorem{algorithm}[theorem]{Algorithm}
\newtheorem{invariant}[theorem]{Invariant}

% Definition of the proof-environment:
\newenvironment{proof}{{\bf Proof:}\quad}{$\Box$}

% Definition of an eqnarray-like environment with an extra
% column for comments
\newcommand{\eeqnarray}[1]{\[\begin{array}{rcll} #1 \end{array}\]}

% Definition of a eqnarray-like environment for proofs of the
% form 
%     init     
%=>   step1   reason1
%=>   step2   reason2
%...
\newcommand{\peqnarray}[1]{\[\begin{array}{cll} #1 \end{array}\]}

%
% An environment for formatting programs. 
%
%	written by Hal Perkins
%	adapted by Charles Elkan	9/3/86
%	keyword macros by Anne Neirynck
%
% Usage :
%
% \begin{program}
% program text\\
% program text
% \end{program}
%
% The program environment is a tabbing environment with ten tab stops spaced
% evenly from the left of the page.  Initially the left margin is the second
% tab stop.  Use \+ to indent following lines one more tab stop, \- to undo
% the effect of \+, and \> at the beginning of a line to indent an extra tab.

\newlength{\pgmtab}          %  \pgmtab is the width of each tab in the
\setlength{\pgmtab}{2em}     %  program environment

% boxed program is like program, only boxed!
% This is useful for centering programs and preventing page breaks in programs.
% argument t or b is required.

\newenvironment{boxed-program}[1]{\begin{minipage}[#1]{9in}
\begin{tabbing}\hspace{\pgmtab}\=\hspace{\pgmtab}\=%
\hspace{\pgmtab}\=\hspace{\pgmtab}\=\hspace{\pgmtab}\=\hspace{\pgmtab}\=%
\hspace{\pgmtab}\=\hspace{\pgmtab}\=\hspace{\pgmtab}\=\hspace{\pgmtab}\=%
\hspace{\pgmtab}\=%
\kill}{\end{tabbing}\end{minipage}}

\newenvironment{program}{\begin{tabbing}\hspace{\pgmtab}\=\hspace{\pgmtab}\=%
\hspace{\pgmtab}\=\hspace{\pgmtab}\=\hspace{\pgmtab}\=\hspace{\pgmtab}\=%
\hspace{\pgmtab}\=\hspace{\pgmtab}\=\hspace{\pgmtab}\=\hspace{\pgmtab}\=%
\hspace{\pgmtab}\=%
\+\+\kill}{\end{tabbing}}

% The following commands should be used OUTSIDE math mode

\newcommand {\FUNCTION}{{\bf function\ }}
\newcommand {\BEGIN}{{\bf begin\ }}
\newcommand {\END}{{\bf end}}
\newcommand {\CASE}{{\bf case}}
\newcommand {\OF}{{\bf of}}
\newcommand {\SELECT}{{\bf select\ }}
\newcommand {\WHERE}{{\bf where\ }}
\newcommand {\DECLARE}{{\bf declare\ }}
\newcommand {\ARRAY}{{\bf array\ }}
\newcommand {\LET}{{\bf\ let\ }}
\newcommand {\IN}{{\bf\ in\ }}
\newcommand {\IF}{{\bf if\ }}
\newcommand {\THEN}{{\bf then\ }}
\newcommand {\ELSE}{{\bf else\ }}
\newcommand {\SKIP}{{\bf skip\ }}
\newcommand {\DO}{{\bf do\ }}
% OLD---don't use \OD
\newcommand {\OD}{{\bf od\ }}
\newcommand {\BY}{{\bf by\ }}
\newcommand {\LOOP}{{\bf loop\ }}
\newcommand {\WHILE}{{\bf while\ }}
\newcommand {\TO}{{\bf to\ }}
\newcommand {\DOWNTO}{{\bf down to\ }}
\newcommand {\FOR}{{\bf for\ }}
\newcommand {\FOREACH}{{\bf for each\ }}
\newcommand {\RETURN}{{\bf return\ }}
\newcommand {\REPEAT}{{\bf repeat\ }}
\newcommand {\UNTIL}{{\bf until\ }}
\newcommand {\LCOM}{$(\ast \;$}
\newcommand {\RCOM}{$\ast)$}
\newcommand {\GOTO}{{\bf goto\ }}

% The following commands are for use INSIDE math mode

\newcommand {\OP}[1]{\mbox{\sc #1}}
\newcommand {\op}[1]{\mbox{\sc #1}}
\newcommand {\mm}[1]{\mbox{\rm #1}\;}
\newcommand {\id}[1]{\mbox{\it #1\ }}

\newcommand {\ASSIGN}{\leftarrow }
\newcommand {\MIN}{\OP{min} }
\newcommand {\MOD}{\; {\bf{\rm mod}} \; } 
\newcommand {\LAMBDA}{{\bf \lambda\ }}
\newcommand {\FALSE}{{\em FALSE\ }}

% Miscellaneous notation

\newcommand {\And}{\wedge}
\newcommand {\Or}{\vee}

\newcommand {\thus}{{\dot{. \: .}\;}}

\newcommand {\bigO}[1]{{\cal O}(#1)}

\newcommand {\app}{\!\!:\!}
\newcommand {\hastype}{::}
\newcommand{\hast}{:}
\newcommand{\qt}[1]{\mbox{``#1''}}
\newcommand{\TexComment}[1]{}

\newcommand{\dq}{\m{\tt "}}
\newcommand{\flatqt}[1]{\m{\dq #1 \dq}}
\newcommand{\seq}{\subseteq}
\newcommand{\derives}{\vdash}

% for writing grammars
\newcommand{\grammar}{::=}
\newcommand{\gor}{\,|\,}

% macros for writing inference rules
\newcommand{\infrule}[2]{\displaystyle{\displaystyle\strut{#1}} \over %
                                        {\displaystyle\strut {#2}}}
\newcommand{\cinfrule}[3]{\parbox{14cm}{\hfil$\infrule{#1}{#2}$\hfil}\parbox{4cm}{$\,#3$\hfil}}

%Let's sqeeze a little more onto a page:
%\textheight 9.1in 

\begin{document}
\handout{\PAOneNum}{5}{Programming Assignment I \\ Due Thursday, October
  9, 2008 at 11:59pm}

%\maketitle

% Three macros for defining:
%       Unix elements: filenames and program (sans serif)
%       Cool elements: literal tokens (typewriter)
%       C elements: function and variable names (boldface)
%
\def\U#1{{\sf{}#1}}
\def\S#1{{\tt{}#1}} % NB: we often use \verb+...+ for this also
\def\C#1{{\bf{}#1}}

\newcommand{\Char}[1]{\ensuremath{\fbox{\phantom{$\backslash$}$\!\!\!$#1}\;}}


\section{Overview}

Programming assignments I--IV will direct you to design and build a
compiler for Cool.  Each assignment will cover one component of the
compiler: lexical analysis, parsing, semantic analysis, and code
generation.  Each assignment will ultimately result in a working
compiler phase which can interface with other phases.  You will have an
option of doing your projects in C++ or Java.

For this assignment, you are to write a lexical analyzer, also called a
{\em scanner}, using a {\em lexical analyzer generator}. (The C++ tool is
called \U{flex}; the Java tool is called \U{jlex}.)  You will describe
the set of tokens for Cool in an appropriate input format, and the
analyzer generator will generate the actual code (C++ or Java) for
recognizing tokens in Cool programs.

On-line documentation for all the tools needed for the project will be
made available on the CS143 web site.  This includes manuals for \U{flex}
and \U{jlex} (used in this assignment), the documentation for \U{bison}
and \U{java\_cup} (used in the next assignment), as well as the manual for
the spim simulator.

You may work either individualy or in pairs for this assignment.

%If you have not done so already, you might consider buying the optional
%reader available for sale from Copy Central on the corner of Euclid and
%Hearst.  In addition to having manuals for \U{flex} and \U{jlex}, the
%reader contains the documentation for \U{bison} and \U{java\_cup}, which
%will be used in the next assignment, as well as the manual for the
%\U{spim} simulator.  All of these materials are also available online
%through the course website.

%You must work in a group for this assignment (where a group consists of
%one or two people).  The \U{submit} program will ask you to specify
%group members when you turn in your assignment.


\section{Files and Directories}

To get started, create a directory where you want to do the assignment
and execute one of the following commands {\em in that directory}.  The
assignments will be graded on the {\em myth} machines, so you should
also work on one of them. For the C++ version of the assignment, you should
type

\begin{verbatim}
gmake -f /usr/class/cs143/assignments/PA2/Makefile
\end{verbatim}

Note that even though this is the first programming assignment, the directory
name is {\em PA2}.  Future assignments will also have directories that are one
more than the assignment number--please don't get confused!  This situation
arises because we are skipping the usual first assignment in this offering
of the course.
\noindent For Java, type:

\begin{verbatim}
gmake -f /usr/class/cs143/assignments/PA2J/Makefile
\end{verbatim}

\noindent
(notice the ``J'' in the path name).  This command will copy a number of
files to your directory.  Some of the files will be copied read-only
(using symbolic links).  You should not edit these files.  In fact, if
you make and modify private copies of these files, you may find it
impossible to complete the assignment.  See the instructions in the
\U{README} file.  The files that you will need to modify are:
\begin{itemize}
\item \U{cool.flex} (in the C++ version) / \U{cool.lex} (in the Java version)\\
This file contains a skeleton for a lexical description for Cool.  There
are comments indicating where you need to fill in code, but this is not
necessarily a complete guide.  Part of the assignment is for you to
make sure that you have a correct and working lexer.  Except for the
sections indicated, you are welcome to make modifications to our
skeleton.  You
can actually build a scanner with the skeleton description, but it does not do
much. You should read the \U{flex}/\U{jlex} manual to figure out what
this description does do.  Any auxiliary routines that you wish to write
should be added directly to this file in the appropriate section (see
comments in the file).

\item \U{test.cl} \\
This file contains some sample input to be scanned. It does not exercise
all of the lexical specification, but it is nevertheless an interesting
test.  It is not a good test to start with, nor does it provide adequate
testing of your scanner.  Part of your assignment is to come up with
good testing inputs and a testing strategy.  (Don't take this
lightly---good test input is difficult to create, and forgetting to test
something is the most likely cause of lost points during grading.)

You should modify this file with tests that you think adequately
exercise your scanner.  Our \U{test.cl} is similar to a real Cool
program, but your tests need not be.  You may keep as much or as little
of our test as you like.

\item \U{README}\\
This file contains detailed instructions for the assignment as well as a
number of useful tips.  You should also edit this file to include the
write-up for your project.  You should explain design decisions, why your
code is correct, and why your test cases are adequate.  It is part of the
assignment to clearly and concisely explain things in text as well as to
comment your code.

\end{itemize}

Although these files are incomplete as given, the lexer does compile
and run (\U{gmake lexer}).

%All of the software supplied with this assignment is supported on the
%Solaris/SPARC and Solaris/x86 machines.  However, if you switch
%platforms (use the \U{arch} command to determine what platform you're
%on) be sure to run \U{gmake clean} to remove files compiled for the
%other architecture.  A version of the project for Linux is available
%for downloading on the course web page.

\section{Scanner Results}\label{sec:results}

You should follow the specification of the lexical structure of Cool
given in Section~10 and Figure 1 of the Cool manual.  Your scanner
should be robust---it should work for any conceivable input.  For
example, you must handle errors such as an EOF occurring in the middle
of a string or comment, as well as string constants that are too long.
These are just some of the errors that can occur; see the manual for
the rest.

You must make some provision for graceful termination if a fatal error
occurs. Core dumps or uncaught exceptions are unacceptable.

\subsection{Error Handling}

\emph{All} errors should be passed along to the parser.  You lexer
should not print anything.  Errors are communicated to the parser by
returning a special error token called \C{ERROR}.  (Note, you should
ignore the token called \C{error} [in lowercase] for this assignment;
it is used by the parser in PA3.)
%The Cool parser knows about
%a special error token called \C{ERROR}, which is used to communicate
%$errors from the lexer to the parser.
There are several requirements for reporting and recovering from
lexical errors:

\begin{itemize}

  \item When an invalid character (one that can't begin any token) is
  encountered, a string containing just that character should be
  returned as the error string.  Resume lexing at the following character.

  \item If a string contains an unescaped newline, report that error as
  \S{``Unterminated string constant''} and resume lexing at the
  beginning of the next line---we assume the programmer simply
  forgot the close-quote.

  \item When a string is too long, report the error as \S{``String
        constant too long''} in the error string in the \C{ERROR}
        token.  If the string contains invalid
        characters (i.e., the null character), report this as
        \S{``String contains null character''}.  In either case,
        lexing should resume after the end of the string.  The
        end of the string is defined as either
        \begin{enumerate}
          \item the beginning of the next line if an unescaped
                newline occurs after these errors are encountered; or
          \item after the closing \C{"} otherwise.
        \end{enumerate}


  \item If a comment remains open when EOF is encountered, report this
  error with the message \S{``EOF in comment''}.  Do \emph{not} tokenize the
  comment's contents simply because the terminator is missing.
  Similarly for strings, if an EOF is encountered before the
  close-quote, report this error as \S{``EOF in string constant''}.

  \item If you see ``{\tt *)}'' outside a comment, report this error as
  \S{``Unmatched *)''}, rather than tokenzing it as {\tt *}
  and {\tt )}.


\end{itemize}


\subsection{String Table}

Programs tend to have many occurrences of the same lexeme.  For example,
an identifier is generally referred to more than once in a program (or
else it isn't very useful!).  To save space and time, a common compiler
practice is to store lexemes in a {\em string table}.  We provide a
string table implementation for both C++ and Java.  See the following
sections for the details.

%When an invalid character is encountered, that
%character and any invalid characters that follow should be gathered
%together into a string until the lexer finds a character that can begin
%a new token.  This string will be returned as the error message.
% sm: I think the above is not a good requirement
%When an invalid character is encountered, a string containing just
%that character should be returned as the error string.
%For
%errors besides strings of invalid characters (e.g., a string constant
%that is too long, or an end-of-file inside of a comment) it is
%sufficient to return an informative error message (e.g., ``String
%constant too long'' or ``EOF in comment'').  Make sure that the error
%message is informative so that we can understand what you did.  The
%following sections clarify how to actually return the error message in
%C++ and Java versions of the assignment.


There is an issue in deciding how to handle the special identifiers for
the basic classes (\C{Object}, \C{Int}, \C{Bool}, \C{String}), {\tt
SELF\_TYPE}, and {\tt self}.  However, this issue doesn't actually come
up until later phases of the compiler---the scanner should treat the
special identifiers exactly like any other identifier.

Do \emph{not} test whether integer literals fit within the
representation specified in the Cool manual---simply create a Symbol
with the entire literal's text as its contents, regardless of its
length.

\subsection{Strings}

Your scanner should convert escape characters in string constants to their
correct values.  For example, if the programmer types these eight
characters:
\begin{center}
{\tt
  \Char{"}\Char{a}\Char{b}\Char{$\backslash$}\Char{n}\Char{c}\Char{d}\Char{"}
}
\end{center}
your scanner would return the token \C{STR$\_$CONST} whose semantic value
is these 5 characters:
\begin{center}
{\tt
  \Char{a}\Char{b}\Char{$\backslash$n}\Char{c}\Char{d}
}
\end{center}
where \Char{$\backslash$n} represents the literal ASCII character for
newline.

Following specification on page 15 of the Cool manual, you must return
an error for a string containing the literal null character. However,
the sequence of two characters
\begin{center}
{\tt\Char{$\backslash$}\Char{0}}
\end{center}
is allowed but should be converted to the one character
\begin{center}
{\tt\Char{0}}.
\end{center}

%In both Flex and JLex, you can produce this code by typing
%$\backslash$n.  

\subsection{Other Notes}

Your scanner should maintain the variable \C{curr\_lineno} that
indicates which line in the source text is currently being scanned.
This feature will aid the parser in printing useful error messages.

You should ignore the token \C{LET\_STMT}.  It is used only by the
parser (PA3).  Finally, note that if the lexical specification is
incomplete (some input has no regular expression that matches), then
the scanners generated by both \U{flex} and \U{jlex} do undesirable
things.  {\em Make sure your specification is complete}.

%Finally, if the lexical specification is incomplete (some input has no
%regular expression that matches) then the generated scanner will invoke
%a default action on unmatched strings.  The default action simply copies
%the string to the console.  Your final scanner should have no default
%actions.  Note that default actions are very bad for \U{mycoolc}, which
%works by piping output from one compiler phase to the next; any extra
%output will cause errors in downstream phases.

\section{Notes for the C++ Version of the Assignment}

If you are working on the Java version, skip to the following section.

\begin{itemize}

\item Each call on the scanner returns the next token and lexeme from the
input.  The value returned by the function \C{cool\_yylex} is an integer
code representing the syntactic category (e.g., integer
literal, semicolon, \S{if} keyword, etc.).  The codes for all tokens
are defined in the file \U{cool-parse.h}.  The second component, the
semantic value or lexeme, is placed in the global union \C{cool\_yylval},
which is of type YYSTYPE.  The type YYSTYPE is also defined in
\U{cool-parse.h}.  The tokens for single character symbols
(e.g., ``;'' and ``,'') are represented just by the
integer (ASCII) value of the character itself.  All of the single character
tokens are listed in the grammar for Cool in the Cool manual.

\item
For class identifiers, object identifiers, integers, and strings, the
semantic value should be a \C{Symbol} stored in the field
\C{cool\_yylval.symbol}.  For boolean constants, the semantic value is
stored in the field \C{cool\_yylval.boolean}. 
Except for errors (see
below), the lexemes for the other tokens do not carry any interesting
information.

\item
We provide you with a string table implementation, which is discussed in
detail in {\em A Tour of the Cool Support Code} and in documentation in the
code.  For the moment, you only need to know that the type of string
table entries is \C{Symbol}.

\item
When a lexical error is encountered, the routine \C{cool\_yylex} should
return the token \C{ERROR}.  The semantic value is the string
representing the error message, which is stored in the field
\C{cool\_yylval.error\_msg} (note that this field is an ordinary string,
not a symbol). See the previous section for information on what to put in
error messages.

\end{itemize}

\section{Notes for the Java Version of the Assignment}

If you are working on the C++ version, skip this section.

\begin{itemize}

\item
Each call on the scanner returns the next token and lexeme from the
input.  The value returned by the method \C{CoolLexer.next\_token} is an
object of class \C{java\_cup.runtime.Symbol}.  This object has a field
representing the syntactic category of a token (e.g.,
integer literal, semicolon, the \S{if} keyword, etc.). The syntactic codes
for all tokens are defined in the file \U{TokenConstants.java}. The
component, the semantic value or lexeme (if any), is also placed in a
\C{java\_cup.runtime.Symbol} object.  The documentation for the class
\C{java\_cup.runtime.Symbol} as well as other supporting code is
available on the course web page.  Examples of its use are also given
in the skeleton.

\item
For class identifiers, object identifiers, integers, and strings, the
semantic value should be of type \C{AbstractSymbol}.  For boolean constants,
the semantic value is of type \C{java.lang.Boolean}. 
Except for errors
(see below), the lexemes for the other tokens do not carry any
interesting information.  Since the \C{value} field of class
\C{java\_cup.runtime.Symbol} has generic type \C{java.lang.Object},
you will need to cast it to a proper type before calling any methods
on it.

\item
We provide you with a string table implementation, which is defined in
\U{AbstractTable.java}. 
For the moment, you only need to know
that the type of string table entries is \C{AbstractSymbol}.

\item
When a lexical error is encountered, the routine
\C{CoolLexer.next\_token} should return a \C{java\_cup.runtime.Symbol} object
whose syntactic category is \C{TokenConstants.ERROR} and whose semantic
value is the error message string.  See Section~\ref{sec:results} for
information on how to construct error messages.

\end{itemize}

\section{Testing the Scanner}

There are at least two ways that you can test your scanner.  The first
way is to generate sample inputs and run them using \U{lexer}, which
prints out the line number and the lexeme of every token recognized by
your scanner.  The other way, when you think your scanner is working,
is to try running \U{mycoolc} to invoke your lexer together with all
other compiler phases (which we provide).  This will be a complete
Cool compiler that you can try on any test programs.

\section{What to Turn In}

When you are ready to turn in the assignment, type \U{gmake
submit-clean} in the directory where you have prepared your assignment.
This action will remove all the unnecessary files, such as object files,
class files, core dumps, Emacs autosave files, etc.  Following \U{gmake
submit-clean}, type \U{/usr/class/cs143/bin/submit} and follow the
directions. Make sure you have properly edited the README file before
submitting your work.

%Doctoring the output that is sent is considered cheating (and not
%effective, since we test your program ourselves).  If you want to
%explain something, do it in the \U{README} file.

The last submission you do will be the one graded.  Each submission overwrites
the previous one.  Remember that there is a 0.5\% penalty per hour for
late assignments.  The burden of convincing us that you understand the
material is on you.  Obtuse code, output, and write-ups will have a
negative effect on your grade. Take the extra time to clearly (and
concisely!) explain your results.

\end{document}



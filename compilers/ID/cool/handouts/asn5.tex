\documentclass[11pt]{article}
\usepackage{latexsym}
\usepackage{handout}
%
% Copyright (c) 1995-1996 by Alex Aiken.  All rights reserved.
% Permission is granted to modify and distribute this document for
% for non-commercial purposes, so long as this copyright notice is retained
% in all copies.
%
% Side margins:
% Actual margin is 1 in + this number
\oddsidemargin -0.25in
\evensidemargin -0.25in

% Text width:
\textwidth 6.9in

% Top margin:
% Actual margin is 1.5 in + this number
\topmargin -.3in

% Text height:
\textheight 8.7in


% Handout Numbers

\newcommand{\PAOneNum}{1}
\newcommand{\WAOneNum}{2}
\newcommand{\PATwoNum}{3}
\newcommand{\PAThreeNum}{4}
\newcommand{\WATwoNum}{5}
\newcommand{\WATwoSolNum}{6}
\newcommand{\WAOneSolNum}{7}
\newcommand{\WAThreeNum}{8}
\newcommand{\WAFourNum}{9}
\newcommand{\WAThreeSolNum}{10}
\newcommand{\PAFourNum}{11}
\newcommand{\WAFourSolNum}{12}
\newcommand{\WAFiveNum}{13}
\newcommand{\WAFiveSolNum}{14}
\newcommand{\WASixNum}{15}
\newcommand{\WASixSolNum}{16}
\newcommand{\PAFiveNum}{17}
\newcommand{\WASevenNum}{18}
\newcommand{\WASevenSolNum}{19}
\newcommand{\PAExtraNum}{20}
\newcommand{\WAEightNum}{21}
\newcommand{\WAEightSolNum}{22}
\newcommand{\PASixNum}{23}
\newcommand{\WANineNum}{24}
\newcommand{\WANineSolNum}{25}
\newcommand{\WATenNum}{26}
\newcommand{\WATenSolNum}{27}
\newcommand{\WAElevenNum}{28}
\newcommand{\WAElevenSolNum}{29}


\newcommand{\attr}[3]{#1:#2\leftarrow#3}
\newcommand{\classmap}[2]{class(#1) = (#2)}
\newcommand{\ossimple}[6]{#1,#2,#3\vdash #4 : #5,#6}
\newcommand{\osrule}[8]{\frac{#7}{\ossimple{#1}{#2}{#3}{#4}{#5}{#6}}\eqno
\mbox{#8}}
\def\U#1{{\sf{}#1}}
\def\S#1{{\tt{}#1}} % NB: we often use \verb+...+ for this also
\def\C#1{{\bf{}#1}}
\newcommand{\dq}{\mbox{\tt "}}

\begin{document}

\handout{7}{6}{Programming Assignment IV \\
Due Thursday, March 8, 2007 at 11:59 PM }


\section{Introduction}

In this assignment, you will implement a code generator for Cool.
When successfully completed, you will have a fully functional Cool
compiler!

The code generator makes use of the AST constructed in PA3 and static
analysis performed in PA4.  Your code generator should produce MIPS
assembly code that faithfully implements {\em any} correct Cool
program.  There is no error recovery in code generation---all
erroneous Cool programs have been detected by the front-end phases of
the compiler.

As with the static analysis assignment, this assignment has
considerable room for design decisions.  Your program is correct if
the code it generates works correctly; how you achieve that goal is up
to you.  We will suggest certain conventions that we believe will make
your life easier, but you do not have to take our advice.  As always,
explain and justify your design decisions in the \U{README} file.
This assignment is about twice the amount of the code of the previous
programming assignment, though they share much of the same
infrastructure. {\bf Start early!}

Critical to getting a correct code generator is a thorough
understanding of both the expected behavior of Cool constructs and the
interface between the runtime system and the generated code.  The
expected behavior of Cool programs is defined by the operational
semantics for Cool given in Section 13 of the \emph{Cool Reference
Manual}.  Recall that this is only a specification of the meaning of
the language constructs---not how to implement them.  The interface
between the runtime system and the generated code is given in
\emph{The Cool Runtime System}.  See that document for a detailed
discussion of the requirements of the runtime system on the generated
code.  There is a lot of information in this handout and the
aforementioned documents, and you need to know most of it to write a
correct code generator.  {\em Please read thoroughly.}

You may work in a group of one or two people.

\section{Files and Directories}

To get started, create a directory where you want to do the assignment
and execute one of the following commands {\em in that directory.}
If you are using C++, run this command to get the initial
skeleton:
\begin{verbatim}
  gmake -f /usr/class/cs143/assignments/PA5/Makefile
\end{verbatim}
For Java, type:
\begin{verbatim}
  gmake -f /usr/class/cs143/assignments/PA5J/Makefile
\end{verbatim}
(notice the ``J'' in the path name).

As usual, there are other files used in the assignment that are
symbolically linked to your directory or are included from \U{/usr/class/cs143/include}
and \U{/usr/class/cs143/src}.  You should not modify these
files.  Almost all of these files have been described in previous
assignments.

We now describe the most important files for each version of the
project.

\subsection{C++ Version}

This is a list of the files that you may want to modify.
You should already be familiar with most of the other files from previous assignments.
See the \U{README} file for details about the additional
files.


\begin{itemize}

\item \U{cgen.cc} \\
This file will contain almost all your code for the code generator.
The entry point for your code generator is the
\C{program\_class::cgen(ostream\&)} method, which is called on the
root of your AST.  Along with the usual constants, we have provided
functions for emitting MIPS instructions, a skeleton for coding
strings, integers, and booleans, and a skeleton of a class table
(\C{CgenClassTable}).  You can use the provided code or replace it
with your own inheritance graph from PA4.

\item \U{cgen.h} \\
This file is the header for the code generator.  You may add anything you
like to this file.  It provides classes for implementing the inheritance
graph.  You may replace or modify them as you wish.

\item \U{emit.h} \\
This file contains various code generation macros used in emitting MIPS
instructions among other things.  You may modify this file.

\item \U{cool-tree.h} \\
As usual, these files contain the declarations of classes for AST
nodes.  You can add field or method declarations to the classes in
\U{cool-tree.h}.  The implementation of methods should be added to
\U{cgen.cc}.

\item \U{cgen\_supp.cc} \\
This file contains general support code for the code generator.  You
will find a number of handy functions here.  Add to the file as you see
fit, but don't change anything that's already there.

\item \U{example.cl} \\
This file should contain a test program of your own design.  
Test as many features of the code generator as you can.

\item \U{README}\\
This file will contain the write-up for your assignment. 
It is critical that you explain design decisions, how your
code is structured, and why you believe your design is a good one
(i.e., why it leads to a correct and robust program).  It is part of
the assignment to explain things in text as well as to comment your
code. 

\end{itemize}



\subsection{Java Version}

This is a list of the files that you may want to modify.
You should already be familiar with most of the other files from previous assignments.
See the \U{README} file for details about the additional
files.

\begin{itemize}

\item \U{CgenClassTable.java} and \U{CgenNode.java} \\
These files provide an implementation of the inheritance graph for the
code generator.  You will need to complete CgenClassTable in order to
build your code generator. You can use the provided code or replace it
with your own inheritance graph from PA4.

\item \U{StringSymbol.java}, \U{IntSymbol.java}, and \U{BoolConst.java} \\
These files provide support for Cool constants.  You
will need to complete the method for generating constant definitions.

\item \U{cool-tree.java} \\
This file contains the definitions for the AST nodes.  You will need
to add code generation routines (\C{code(PrintStream)}) for Cool
expressions in this file.  The code generator is invoked by calling
method \C{cgen(PrintStream)} of class \C{program}. You may add new
methods, but do not modify the existing declarations.

\item \U{TreeConstants.java} \\
As before, this file defines some useful symbol constants. Feel free to
add your own as you see fit.

\item \U{CgenSupport.java} \\
This file contains general support code for the code generator.  You
will find a number of handy functions here including ones for emitting
MIPS instructions.  Add to the file as you see fit, but don't change
anything that's already there.

\item \U{example.cl} \\
This file should contain a test program of your own design.  
Test as many features of the code generator as you can.


\item \U{README} \\
This file will contain the write-up for your assignment. 
It is critical that you explain design decisions, how your
code is structured, and why you believe your design is a good one
(i.e., why it leads to a correct and robust program).  It is part of
the assignment to explain things in text as well as to comment your
code. 

\end{itemize}



\section{Design}

Before continuing, we suggest you read \emph{The Cool Runtime System}
to familiarize yourself with the requirements on your code generator
imposed by the runtime system.

In considering your design, at a high-level, your code generator will
need to perform the following tasks:
\begin{enumerate}
  \item Determine and emit code for global constants, such as
        prototype objects.
  \item Determine and emit code for global tables, such as the
        \C{class\_nameTab}, the \C{class\_objTab}, and the dispatch
        tables.
  \item Determine and emit code for initialization method for
        each class.
  \item Determine and emit code for each method definition.
\end{enumerate}

There are many possible ways to write the code generator.  One
reasonable strategy is to perform code generation in two passes.  The
first pass decides the object layout for each class, particularly the
offset at which each attribute is stored in an object.  Using this
information, the second pass recursively walks each feature and
generates stack machine code for each expression.

There are a number of things you must keep in mind while designing
your code generator:
\begin{itemize}

\item Your code generator must work correctly with the Cool runtime
system, which is explained in the {\em Cool Runtime System} manual.

\item You should have a clear picture of the runtime semantics of Cool
programs.  The semantics are described informally in the first part of
the \emph{Cool Reference Manual}, and a precise description of how
Cool programs should behave is given in Section 13 of the manual.

\item You should understand the MIPS instruction set.  An overview of
MIPS operations is given in the \U{spim} documentation, which is
on the class web page.

\item You should decide what invariants your generated code will
observe and expect (i.e., what registers will be saved, which might be
overwritten, etc).  You may also find it useful to refer to information
on code generation in the lecture notes.
\end{itemize}

You do {\em not} need to generate the same code as \U{coolc}.
\U{Coolc} includes a very simple register allocator and other small
changes that are not required for this assignment. The only
requirement is to generate code that runs correctly with the runtime
system.

\subsection{Runtime Error Checking}
The end of the Cool manual lists six errors that will terminate the
program.  Of these, your generated code should catch the first
three---dispatch on void, case on void, and missing branch---and print
a suitable error message before aborting.  You may allow SPIM to catch
division by zero.  Catching the last two errors---substring out of
range and heap overflow---is the responsibility of the runtime system
in {\tt trap.handler}.  See Figure 4 of the \emph{Cool Runtime System}
manual for a listing of functions that display error messages for you.

\subsection{Garbage Collection}

To receive full credit for this assignment, your code generator must
work correctly with the generational garbage collector in the Cool
runtime system.  The skeletons contain functions \C{code\_select\_gc}
(C++) and \C{CgenClassTable.codeSelectGc} (Java) that generate code
that sets GC options from command line flags.  The command-line flags
that affect garbage collection are \U{-g}, \U{-t}, and \U{-T}.
Garbage collection is disabled by default; the flag \U{-g} enables it.
When enabled, the garbage collector not only reclaims memory, but also
verifies that ``-1'' separates all objects in the heap, thus checking
that the program (or the collector!) has not accidentally overwritten
the end of an object.  The \U{-t} and \U{-T} flags are used for
additional testing.  With \U{-t} the collector performs collections
very frequently (on every allocation).  The garbage collector does not
directly use \U{-T}; in \U{coolc} the \U{-T} option causes extra code
to be generated that performs more runtime validity checks.  You are
free to use (or not use) \U{-T} for whatever you wish.

For your implementation, the simplest way to start is not to use the
collector at all (this is the default).  When you decide to use the
collector, be sure to carefully review the garbage collection
interface described in the {\em Cool Runtime System} manual.  Ensuring
that your code generator correctly works with the garbage collector in
{\em all} circumstances is not trivial.



\section{Testing and Debugging}

You will need a working scanner, parser, and semantic analyzer to test
your code generator. You may use either your own components or the
components from \U{coolc}.  By default, the \U{coolc} components are
used.  To change that, replace the \U{lexer}, \U{parser}, and/or
\U{semant} executable (which are symbolic links in your project
directory) with your own scanner/parser/semantic analyzer.  Even if you use your own
components, it is wise to test your code generator with the \U{coolc}
scanner, parser, and semantic analyzer at least once because we will
grade your project using \U{coolc}'s version of the other phases.

You will run your code generator using \U{mycoolc}, a shell script
that ``glues'' together the generator with the rest of compiler
phases.  Note that \U{mycoolc} takes a \U{-c} flag for debugging the
code generator; using this flag merely causes {\tt cgen\_debug} (a
global variable in the C++ version and a static field of class {\tt
Flags} in the Java version) to be set.  Adding the actual code to
produce useful debugging information is up to you.  See the project
\U{README} for details.

\subsection{Coolaid}

\newcommand{\Coolaid}{\U{Coolaid}}

\Coolaid{} is a tool to statically verify some basic correctness
properties of the MIPS assembly code produced from Cool
source. \Coolaid{} will check that the assembly code is ``well-typed''
with respect to the Cool typing rules, just like the Java bytecode
verifier checks that the bytecode output by a Java compiler is type
safe. Aside from checking the safety of the output code, this tool can
greatly benefit the development process of the code generation
phase. Since the compiler front-end ensures that the source program is
well-typed and that the type system guarantees certain safety
properties, we expect that those properties should also hold on the
resulting assembly code. If \Coolaid{} is not able to verify some
particular safety property, then a likely cause is a bug in the
compiler itself.

\Coolaid{} has been tested on 3500 programs generated by student
compilers from the Spring 2003 offering of CS164 at UC Berkeley. The
standard testing procedure (which runs the generated code on the MIPS
simulator \U{spim}) used for grading found errors in 1000 of
those. Often the error messages were in the form of garbled output or
output that did not match the expected output, whereas \Coolaid{} was
able to give precise error messages where the unsafe operation was
performed that most likely caused the garbled output.  \Coolaid{}
also finds errors in 800 of the 2500 programs that pass the testing
procedure! These were errors that the testing procedure missed because
the sample input for the code did not exercise the bad code portions.
See the \emph{Coolaid Reference Manual} to get started using
\Coolaid{}.  Note \Coolaid{} imposes some additional requirements on
the generated code to be able to check it; these requirements are also
discussed in the \emph{Cool Runtime System} manual.

\subsection{Spim and XSpim}

The executables \U{spim} and \U{xspim} are simulators for MIPS
architecture on which you can run your generated code.  The program
\U{xspim} works like \U{spim} in that it lets you run MIPS assembly
programs. However, it has many features that allow you to examine the
virtual machine's state, including the memory locations, registers,
data segment, and code segment of the program. You can also set
breakpoints and single step your program.  The documentation for
\U{spim}/\U{xspim} is on the course web page.

\paragraph{Warning.} One thing that makes debugging with \U{spim} difficult is
that \U{spim} is an interpreter for assembly code and not a true
assembler.  If your code or data definitions refer to undefined
labels, the error shows up only if the executing code actually refers
to such a label. Moreover, an error is reported only for undefined
labels that appear in the code section of your program. If you have
constant data definitions that refer to undefined labels, \U{spim}
won't tell you anything. It will just assume the value 0 for such
undefined labels.

\section{Final Submission}

Make sure to complete the following items before submitting to avoid
any penalties.

\begin{minipage}{0.8\linewidth}
\bigskip
\begin{itemize}
  \item[$\Box$]
    Include your write-up in \U{README}.

  \item[$\Box$]
    Include your test cases that test your code generator
    in \U{example.cl}.

  \item[$\Box$]
    Make sure all your code for the code generator is in
    \begin{itemize}
    \item \U{cool-tree.h}, \U{cgen.h},
          \U{cgen.cc}, \U{cgen\_supp.cc}, and \U{emit.h} for the
          C++ version; or
    \item \U{cool-tree.java}, \U{CgenClassTable.java},
          \U{CgenNode.java}, \U{CgenSupport.java}, \U{BoolConst.java},
          \U{IntSymbol.java}, \U{StringSymbol.java},
          \U{TreeConstants.java}, and additional \U{.java} files you
          may have added for the Java version.
    \end{itemize}

\end{itemize}
\end{minipage}

\end{document}

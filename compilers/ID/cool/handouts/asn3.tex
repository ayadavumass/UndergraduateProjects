\documentclass[11pt]{article}
\usepackage{handout}
%
% Copyright (c) 1995-2000 by Alex Aiken.  All rights reserved.
% Permission is granted to modify and distribute this document for
% for non-commercial purposes, so long as this copyright notice is retained
% in all copies.
%
% Side margins:
% Actual margin is 1 in + this number
\oddsidemargin -0.25in
\evensidemargin -0.25in

% Text width:
\textwidth 6.9in

% Top margin:
% Actual margin is 1.5 in + this number
\topmargin -.3in

% Text height:
\textheight 8.7in

% generally useful macros for writing
\newcommand{\TexDir}{/home3/aiken/tex}

% These allow switching interline spacing; the change takes effect immediately:

\makeatletter
\newcommand{\singlespacing}{\let\CS=\@currsize\renewcommand{\baselinestretch}{1}\tiny\CS}
\newcommand{\oneandahalfspacing}{\let\CS=\@currsize\renewcommand{\baselinestretch}{1.25}\tiny\CS}
\newcommand{\doublespacing}{\let\CS=\@currsize\renewcommand{\baselinestretch}{1.5}\tiny\CS}
%setspacingto sets the interline spacing to the value of its argument
% e.g., \setspacingto{1.5} is the same as \doublespacing
\newcommand{\setspacingto}[1]{\let\CS=\@currsize\renewcommand{\baselinestretch}{#1}\tiny\CS}
\makeatother

% nonumber
\newcommand{\nn}{\nonumber}

% Tab for hand-formatting:
\newcommand{\tab}{\hspace*{2em}}

% Angle brackets:
\newcommand{\la}{\langle}
\newcommand{\ra}{\rangle}

% s.t.
\newcommand{\st}{\mbox{\ s.t.\ }}

% otherwise
\newcommand{\ow}{\m{\rm otherwise}}

% if
\newcommand{\mif}{\m{\rm if\ }}

% Harpoons
\newcommand{\rh}{\rightharpoonup}
\newcommand{\lh}{\leftharpoonup}

% Denotational-semantics-style brackets and bottom:
\newcommand{\lbk}{\lbrack\!\lbrack}
\newcommand{\rbk}{\rbrack\!\rbrack}
\newcommand{\bottom}{\perp}

% Projection operator
\newcommand{\proj}{\!\downarrow\!}

% macros for mbox combined with another style
% (useful for changing typefaces in math mode)
\newcommand {\mboxbf}[1]{\mbox{{\bf #1}}}
\newcommand {\mboxit}[1]{\mbox{{\it #1}}}
\newcommand {\mboxem}[1]{\mbox{{\em #1}}}
\newcommand {\m}{\mbox}
\newcommand {\ch}{\rm}
	
% macro for creating a binary operator
%
% example:  \makebinop{\makebinop{\mybmod}{mod}
%	(duplicates the \bmod macro)
%
\def\makebinop#1#2{\def#1{\mskip-\medmuskip \mskip5mu
\mathbin{\rm #2} \penalty900 \mskip5mu \mskip-\medmuskip}}

% Environments for theorems, lemmas, etc.
\newtheorem{theorem}{Theorem}[section]
\newtheorem{lemma}[theorem]{Lemma}
\newtheorem{corollary}[theorem]{Corollary}
\newtheorem{definition}[theorem]{Definition}
\newtheorem{observation}[theorem]{Observation}
\newtheorem{fact}{Fact}[theorem]
\newtheorem{proposition}[theorem]{Proposition}
\newtheorem{example}[theorem]{Example}
\newtheorem{constraint}[theorem]{Constraint}
\newtheorem{axiom}[theorem]{Axiom}
\newtheorem{law}[theorem]{Law}
\newtheorem{algorithm}[theorem]{Algorithm}
\newtheorem{invariant}[theorem]{Invariant}

% Definition of the proof-environment:
\newenvironment{proof}{{\bf Proof:}\quad}{$\Box$}

% Definition of an eqnarray-like environment with an extra
% column for comments
\newcommand{\eeqnarray}[1]{\[\begin{array}{rcll} #1 \end{array}\]}

% Definition of a eqnarray-like environment for proofs of the
% form 
%     init     
%=>   step1   reason1
%=>   step2   reason2
%...
\newcommand{\peqnarray}[1]{\[\begin{array}{cll} #1 \end{array}\]}

%
% An environment for formatting programs. 
%
%	written by Hal Perkins
%	adapted by Charles Elkan	9/3/86
%	keyword macros by Anne Neirynck
%
% Usage :
%
% \begin{program}
% program text\\
% program text
% \end{program}
%
% The program environment is a tabbing environment with ten tab stops spaced
% evenly from the left of the page.  Initially the left margin is the second
% tab stop.  Use \+ to indent following lines one more tab stop, \- to undo
% the effect of \+, and \> at the beginning of a line to indent an extra tab.

\newlength{\pgmtab}          %  \pgmtab is the width of each tab in the
\setlength{\pgmtab}{2em}     %  program environment

% boxed program is like program, only boxed!
% This is useful for centering programs and preventing page breaks in programs.
% argument t or b is required.

\newenvironment{boxed-program}[1]{\begin{minipage}[#1]{9in}
\begin{tabbing}\hspace{\pgmtab}\=\hspace{\pgmtab}\=%
\hspace{\pgmtab}\=\hspace{\pgmtab}\=\hspace{\pgmtab}\=\hspace{\pgmtab}\=%
\hspace{\pgmtab}\=\hspace{\pgmtab}\=\hspace{\pgmtab}\=\hspace{\pgmtab}\=%
\hspace{\pgmtab}\=%
\kill}{\end{tabbing}\end{minipage}}

\newenvironment{program}{\begin{tabbing}\hspace{\pgmtab}\=\hspace{\pgmtab}\=%
\hspace{\pgmtab}\=\hspace{\pgmtab}\=\hspace{\pgmtab}\=\hspace{\pgmtab}\=%
\hspace{\pgmtab}\=\hspace{\pgmtab}\=\hspace{\pgmtab}\=\hspace{\pgmtab}\=%
\hspace{\pgmtab}\=%
\+\+\kill}{\end{tabbing}}

% The following commands should be used OUTSIDE math mode

\newcommand {\FUNCTION}{{\bf function\ }}
\newcommand {\BEGIN}{{\bf begin\ }}
\newcommand {\END}{{\bf end}}
\newcommand {\CASE}{{\bf case}}
\newcommand {\OF}{{\bf of}}
\newcommand {\SELECT}{{\bf select\ }}
\newcommand {\WHERE}{{\bf where\ }}
\newcommand {\DECLARE}{{\bf declare\ }}
\newcommand {\ARRAY}{{\bf array\ }}
\newcommand {\LET}{{\bf\ let\ }}
\newcommand {\IN}{{\bf\ in\ }}
\newcommand {\IF}{{\bf if\ }}
\newcommand {\THEN}{{\bf then\ }}
\newcommand {\ELSE}{{\bf else\ }}
\newcommand {\SKIP}{{\bf skip\ }}
\newcommand {\DO}{{\bf do\ }}
% OLD---don't use \OD
\newcommand {\OD}{{\bf od\ }}
\newcommand {\BY}{{\bf by\ }}
\newcommand {\LOOP}{{\bf loop\ }}
\newcommand {\WHILE}{{\bf while\ }}
\newcommand {\TO}{{\bf to\ }}
\newcommand {\DOWNTO}{{\bf down to\ }}
\newcommand {\FOR}{{\bf for\ }}
\newcommand {\FOREACH}{{\bf for each\ }}
\newcommand {\RETURN}{{\bf return\ }}
\newcommand {\REPEAT}{{\bf repeat\ }}
\newcommand {\UNTIL}{{\bf until\ }}
\newcommand {\LCOM}{$(\ast \;$}
\newcommand {\RCOM}{$\ast)$}
\newcommand {\GOTO}{{\bf goto\ }}

% The following commands are for use INSIDE math mode

\newcommand {\OP}[1]{\mbox{\sc #1}}
\newcommand {\op}[1]{\mbox{\sc #1}}
\newcommand {\mm}[1]{\mbox{\rm #1}\;}
\newcommand {\id}[1]{\mbox{\it #1\ }}

\newcommand {\ASSIGN}{\leftarrow }
\newcommand {\MIN}{\OP{min} }
\newcommand {\MOD}{\; {\bf{\rm mod}} \; } 
\newcommand {\LAMBDA}{{\bf \lambda\ }}
\newcommand {\FALSE}{{\em FALSE\ }}

% Miscellaneous notation

\newcommand {\And}{\wedge}
\newcommand {\Or}{\vee}

\newcommand {\thus}{{\dot{. \: .}\;}}

\newcommand {\bigO}[1]{{\cal O}(#1)}

\newcommand {\app}{\!\!:\!}
\newcommand {\hastype}{::}
\newcommand{\hast}{:}
\newcommand{\qt}[1]{\mbox{``#1''}}
\newcommand{\TexComment}[1]{}

\newcommand{\dq}{\m{\tt "}}
\newcommand{\flatqt}[1]{\m{\dq #1 \dq}}
\newcommand{\seq}{\subseteq}
\newcommand{\derives}{\vdash}

% for writing grammars
\newcommand{\grammar}{::=}
\newcommand{\gor}{\,|\,}

% macros for writing inference rules
\newcommand{\infrule}[2]{\displaystyle{\displaystyle\strut{#1}} \over %
                                        {\displaystyle\strut {#2}}}
\newcommand{\cinfrule}[3]{\parbox{14cm}{\hfil$\infrule{#1}{#2}$\hfil}\parbox{4cm}{$\,#3$\hfil}}

% Handout Numbers

\newcommand{\PAOneNum}{1}
\newcommand{\WAOneNum}{2}
\newcommand{\PATwoNum}{3}
\newcommand{\PAThreeNum}{4}
\newcommand{\WATwoNum}{5}
\newcommand{\WATwoSolNum}{6}
\newcommand{\WAOneSolNum}{7}
\newcommand{\WAThreeNum}{8}
\newcommand{\WAFourNum}{9}
\newcommand{\WAThreeSolNum}{10}
\newcommand{\PAFourNum}{11}
\newcommand{\WAFourSolNum}{12}
\newcommand{\WAFiveNum}{13}
\newcommand{\WAFiveSolNum}{14}
\newcommand{\WASixNum}{15}
\newcommand{\WASixSolNum}{16}
\newcommand{\PAFiveNum}{17}
\newcommand{\WASevenNum}{18}
\newcommand{\WASevenSolNum}{19}
\newcommand{\PAExtraNum}{20}
\newcommand{\WAEightNum}{21}
\newcommand{\WAEightSolNum}{22}
\newcommand{\PASixNum}{23}
\newcommand{\WANineNum}{24}
\newcommand{\WANineSolNum}{25}
\newcommand{\WATenNum}{26}
\newcommand{\WATenSolNum}{27}
\newcommand{\WAElevenNum}{28}
\newcommand{\WAElevenSolNum}{29}

\begin{document}
\handout{3}{4}{Programming Assignment II \\ 
Due Tuesday, October 21, 2008 at 11:59 PM}

% Three macros for defining:
%       Unix elements: filenames and program (sans serif)
%       Cool elements: literal tokens (typewriter)
%       C elements: function and variable names (boldface)
%
\def\U#1{{\sf{}#1}}
\def\S#1{{\tt{}#1}} % NB: we often use \verb+...+ for this also
\def\C#1{{\bf{}#1}}

\section{Introduction}

In this assignment you will write a parser for Cool. The assignment
makes use of two tools: the parser generator (the C++ tool is called
\U{bison}; the Java tool is called \U{CUP}) and a package for
manipulating trees. The output of your parser will be an abstract syntax
tree (AST). You will construct this AST using semantic actions of the
parser generator.

You certainly will need to refer to the syntactic structure of Cool,
found in Figure 1 of The Cool Reference Manual (as well as other parts).  
The documentation for \U{bison} and \U{CUP} is available
online.  The C++ version of the
tree package is described in the {\em Tour of Cool Support Code}
handout.  The documentation for the Java version is available online.
You will need the tree
package information for this and future assignments.

There is a lot of information in this handout, and you need to know
most of it to write a working parser.  {\em Please read the handout
thoroughly.}

You must work in a group for this assignment (where a group consists of
one or two people).

\section {Files and Directories}

To get started, create a directory where you want to do the assignment
and execute one of the following commands {\em in that directory}.  For
the C++ version of the assignment, you should type

\begin{verbatim}
gmake -f /usr/class/cs143/assignments/PA3/Makefile
\end{verbatim}

\noindent For Java, type:

\begin{verbatim}
gmake -f /usr/class/cs143/assignments/PA3J/Makefile
\end{verbatim}

\noindent
(notice the ``J'' in the path name).  This command will copy a number of
files to your directory.  Some of the files will be copied read-only
(using symbolic links).  You should not edit these files.  In fact, if
you make and modify private copies of these files, you may find it
impossible to complete the assignment.  See the instructions in the
\U{README} file.  The files that you will need to modify are:

\begin{itemize}

\item \U{cool.y} (in the C++ version) / \U{cool.cup} (in the Java version) \\

This file contains a start towards a parser description for Cool.  The
declaration section is mostly complete, but you will need to add
additional type declarations for new nonterminals you introduce.  We
have given you names and type declarations for the terminals.  You
might also need to add precedence declarations.  The rule section,
however, is rather incomplete.  We have provided some parts of some
rules.  You should not need to modify this code to get a working
solution, but you are welcome to if you like.  However, do not assume
that any particular rule is complete.

\item \U{good.cl} and \U{bad.cl} \\
These files test a few features of the grammar.  You should add tests
to ensure that \U{good.cl} exercises every legal construction of the
grammar and that \U{bad.cl} exercises as many types of parsing errors
as possible in a single file.  Explain your tests in these files and
put any overall comments in the \U{README} file.

\item \U{README} \\
As usual, this file will contain the write-up for your assignment.
Explain your design decisions, your test cases, and why you believe
your program is correct and robust.  It is part of the assignment to
explain things in text, as well as to comment your code. 

\end{itemize}

To submit your assignment, run ``{\tt /usr/class/cs143/bin/submit}'' from your PA3
directory and follow the instructions.


\section{Testing the Parser}

You will need a working scanner to test the parser.  You may use either
your own scanner or the \U{coolc} scanner.  By default, the \U{coolc}
scanner is used; to change this behavior, replace the \U{lexer}
executable (which
is a symbolic link in your project directory) with your own scanner.
Don't automatically assume that the scanner (whichever one you use!)  is
bug free---latent bugs in the scanner may cause mysterious problems in
the parser.

You will run your parser using \U{myparser}, a shell script that
``glues'' together the parser with the scanner.  Note that \U{myparser}
takes a \U{-p} flag for debugging the parser; using this flag causes
lots of information about what the parser is doing to be printed on
\S{stdout}.  Both \U{bison} and \U{CUP} produce a human-readable dump of
the LALR(1) parsing tables in the \U{cool.output} file.  Examining this
dump is frequently useful for debugging the parser definition.

%matth: we're not providing coolc and mycoolc now.
%% Once you are confident that your parser is working, try running
%% \U{mycoolc} to invoke your parser together with other compiler
%% phases.  
You should test this compiler on both good and bad inputs to
see if everything is working.  Remember, bugs in your parser may
manifest themselves anywhere.

Your parser will be graded using our lexical analyzer.  Thus, even if
you do most of the work using your own scanner you should test your
parser with the \U{coolc} scanner before turning in the assignment.

\section{Parser Output}

Your semantic actions should build an AST.  The root (and only the root)
of the AST should be of type \S{program}.  For programs that parse
successfully, the output of \S{parser} is a listing of the AST.

For programs that have errors, the output is the error messages of the
parser. We have supplied you with an error reporting routine that prints
error messages in a standard format; please do not modify it. You should
not invoke this routine directly in the semantic actions;
\U{bison}/\U{CUP} automatically invokes it when a problem is detected.

Your parser need only work for programs contained in a single
file---don't worry about compiling multiple files.

\section{Error Handling}

You should use the \S{error} pseudo-nonterminal to add error handling
capabilities in the parser.  The purpose of \S{error} is to permit the
parser to continue after some anticipated error.  It is not a panacea
and the parser may become completely confused.  See the
\U{bison}/\U{CUP} documentation for how best to use \S{error}.  In
your \U{README}, describe which errors you attempt to catch.  Your
test file \U{bad.cl} should have some instances that illustrate the
errors from which your parser can recover.  To receive full credit,
your parser should recover in at least the following situations:

\begin{itemize}

\item If there is an error in a class definition but the class is
terminated properly and the next class is syntactically correct, the
parser should be able to restart at the next class definition.

\item Similarly, the parser should recover from errors in features
(going on to the next feature), a {\tt let} binding (going on to the
next variable), and an expression inside a {\tt \{\ldots\} } block.

\end{itemize}

Do not be overly concerned about the the line numbers that appear in
the error messages your parser generates.  If your parser is working
correctly, the line number will generally be the line where the error
occurred.  For erroneous constructs broken across multiple lines, the
line number will probably be the last line of the construct.


\section{The Tree Package}
\label{sec-tree}

There is an extensive discussion of the C++ version of the tree
package for Cool abstract syntax trees in the {\em Tour} section of the
Cool documentation.  The documentation for the Java version is available on the
course web page. You will need most of that information to write a
working parser.

\section {Remarks}

You may use precedence declarations, but only for expressions.  Do not
use precedence declarations blindly (i.e., do not respond to a
shift-reduce conflict in your grammar by adding precedence rules until
it goes away).
%If you find yourself making up rules for many things
%other than operators in expressions and for {\tt let}, you are probably
%doing something wrong.

The Cool {\tt let} construct introduces an ambiguity into the language
(try to construct an example if you are not convinced).  The manual
resolves the ambiguity by saying that a {\tt let} expression extends
as far to the right as possible.  The ambiguity will show up in your
parser as a shift-reduce conflict involving the productions for {\tt
let}.

This problem has a simple, but slightly obscure, solution.  We will not
tell you exactly how to solve it, but we will give you a strong hint.
In \S{coolc}, we implemented the resolution of the {\tt let}
shift-reduce conflict by using a \U{bison}/\U{CUP} feature that allows
precedence to be associated with productions (not just operators).  See
the \U{bison}/\U{CUP} documentation for information on how to use this
feature.

Since the \U{mycoolc} compiler uses pipes to communicate from one
stage to the next, any extraneous characters produced by the parser
can cause errors; in particular, the semantic analyzer may not be able
to parse the AST your parser produces.

\section{Notes for the C++ version of the assignment}

If you are working on the Java version, skip to the following section.

\begin{itemize}

\item Running \U{bison} on the initial skeleton file will produce some
      warnings about ``useless noterminals'' and ``useless rules''.
      This is because some of the nonterminals and rules will
      never be used, but these \emph{should} go away when your parser
      is complete.

\item You must declare \U{bison} ``types'' for your non-terminals and
terminals that have attributes.  For example, in the skeleton
\U{cool.y} is the declaration:
\begin{verbatim}
%type <program> program
\end{verbatim}
This declaration says that the non-terminal \S{program} has type \S{<program>}.
The use of the word ``type'' is misleading here; what it really means is that
the attribute for the non-terminal \S{program} is stored in the \S{program}
member of the \S{union} declaration in \U{cool.y}, which has type
\S{Program}.   By specifying the type  
\begin{verbatim}
%type <member_name> X Y Z ...
\end{verbatim}
you instruct \C{bison} that the attributes of non-terminals (or terminals)
{\tt X}, {\tt Y}, and {\tt Z} have a type appropriate for the member
{\tt member\_name} of the union.

All the union members and their types have similar names by design.
It is a coincidence in the example above that the non-terminal
\S{program} has the same name as a union member.

It is critical that you declare the correct types for the attributes
of grammar symbols; failure to do so virtually guarantees that your
parser won't work.  You do not need to declare types for symbols of
your grammar that do not have attributes.

The \U{g++} type checker complains if you use the tree constructors
with the wrong type parameters.  If you ignore the warnings, your
program may crash when the constructor notices that it is being used
incorrectly.  Moreover, \U{bison} may complain if you make type
errors.  Heed any warnings.  Don't be surprised if your program
crashes when \U{bison} or \U{g++} give warning messages.

\end{itemize}

\section{Notes for the Java version of the assignment}

If you are working on the C++ version, skip this section.

\begin{itemize}

\item You must declare \S{CUP} ``types'' for your non-terminals and
terminals that have attributes.  For example, in the skeleton
\U{cool.cup} is the declaration:
\begin{verbatim}
nonterminal program program;
\end{verbatim}
This declaration says that the non-terminal \S{program} has type
\S{program}.

It is critical that you declare the correct types for the attributes of
grammar symbols; failure to do so virtually guarantees that your parser
won't work.  You do not need to declare types for symbols of your
grammar that do not have attributes.

The \U{javac} type checker complains if you use the tree constructors
with the wrong type parameters.  If you fix the errors with frivolous
casts, your program may throw an exception when the constructor notices
that it is being used incorrectly.  Moreover, \U{CUP} may complain if
you make type errors. 

\end{itemize}



\end{document}


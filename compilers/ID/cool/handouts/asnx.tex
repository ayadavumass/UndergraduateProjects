\documentclass[11pt]{article}
\usepackage{handout,latexsym}
%
% Copyright (c) 1995-1996 by Alex Aiken.  All rights reserved.
% Permission is granted to modify and distribute this document for
% for non-commercial purposes, so long as this copyright notice is retained
% in all copies.
%
% Side margins:
% Actual margin is 1 in + this number
\oddsidemargin -0.25in
\evensidemargin -0.25in

% Text width:
\textwidth 6.9in

% Top margin:
% Actual margin is 1.5 in + this number
\topmargin -.3in

% Text height:
\textheight 8.7in


\newcommand{\attr}[3]{#1:#2\leftarrow#3}
\newcommand{\classmap}[2]{class(#1) = (#2)}
\newcommand{\ossimple}[6]{#1,#2,#3\vdash #4 : #5,#6}
\newcommand{\osrule}[8]{\frac{#7}{\ossimple{#1}{#2}{#3}{#4}{#5}{#6}}\eqno
\mbox{#8}}
\def\U#1{{\sf{}#1}}
\def\S#1{{\tt{}#1}} % NB: we often use \verb+...+ for this also
\def\C#1{{\bf{}#1}}
\newcommand{\dq}{\mbox{\tt "}}

% Handout Numbers

\newcommand{\PAOneNum}{1}
\newcommand{\WAOneNum}{2}
\newcommand{\PATwoNum}{3}
\newcommand{\PAThreeNum}{4}
\newcommand{\WATwoNum}{5}
\newcommand{\WATwoSolNum}{6}
\newcommand{\WAOneSolNum}{7}
\newcommand{\WAThreeNum}{8}
\newcommand{\WAFourNum}{9}
\newcommand{\WAThreeSolNum}{10}
\newcommand{\PAFourNum}{11}
\newcommand{\WAFourSolNum}{12}
\newcommand{\WAFiveNum}{13}
\newcommand{\WAFiveSolNum}{14}
\newcommand{\WASixNum}{15}
\newcommand{\WASixSolNum}{16}
\newcommand{\PAFiveNum}{17}
\newcommand{\WASevenNum}{18}
\newcommand{\WASevenSolNum}{19}
\newcommand{\PAExtraNum}{20}
\newcommand{\WAEightNum}{21}
\newcommand{\WAEightSolNum}{22}
\newcommand{\PASixNum}{23}
\newcommand{\WANineNum}{24}
\newcommand{\WANineSolNum}{25}
\newcommand{\WATenNum}{26}
\newcommand{\WATenSolNum}{27}
\newcommand{\WAElevenNum}{28}
\newcommand{\WAElevenSolNum}{29}


\begin{document}

\handout{8}{2}{Extra Credit Programming Assignment \\
Due Thursday, March 15, 2007 at 11:59pm}

\section{Introduction} 

This extra credit programming assignment is intended to be a fun
project with which to end the quarter.  There are two choices for this
extra credit assignment:
\begin{quote}
\begin{enumerate}
  \item implement some optimizations in your compiler; or
  \item write a Cool program.
\end{enumerate}
\end{quote}
You may submit either an optimizer or a test program, but not both.

You may work in a group of one or two people.

\section{Optimizer Project}

Extra credit will be awarded for projects that, in addition to code
generation, perform some significant optimization of the code.  The
amount of extra credit depends on how well the optimization is
written, documented, and demonstrated.  Two critical factors are
\begin{quote}
\begin{enumerate}
\item correctness (the optimizations don't result in incorrect programs);
      and
\item the percentage speed-up your optimized code achieves over
      \U{coolc}, as measured by a weighted sum of the instructions
      executed on \U{spim} over a suite of benchmarks of our choosing.
\end{enumerate}
\end{quote}
To find out how many instructions a Cool program executes, run
\U{spim} with the {\tt -keepstats} option.

There are many possible optimizations to implement; see the ASU
chapters 9 and 10 for ideas.  Assuming your initial code generator is
straightforward (like \U{coolc}'s), then two directions that may yield
significant improvement are (1) improving register usage and (2)
specializing the implementation of the basic classes \C{Int} and
\C{String}.

We have not implemented an optimization phase in \U{coolc}, so we have
no skeleton code to give you---you are on your own.  If you want to do
an optimization phase, you are encouraged to talk it over with one of
the course staff first. {\em Under absolutely no circumstances should
you try optimization before your code generator is finished!!}

There is a {\tt -O} flag that controls the global variable
\C{cgen\_optimize} (C++) and \C{Flags.cgen\_optimize} (Java).  You may
use this flag to switch between generating normal code and optimized
code.  For this project, we will always run your compiler with the
{\tt -O} flag on.
%If you do an optimization phase, it should have no effect unless
%\C{cgen\_optimize} is on.
%We will grade your code generator first with
%optimization off; this will prevent you from losing points due to bugs
%in your optimizer.

The total extra credit for doing optimization will not exceed 6\% of
the total grade for the course.  Roughly speaking, the extra credit is
worth up to about half of one of the two large programming
assignments.

\section{Test Program Project}

Extra credit will be awarded for submitted Cool programs that reveal
bugs or strange behavior in optimizing Cool compilers.  The more
projects your program breaks, the more it is worth!  Test cases cannot
be based, even loosely, on pre-existing Cool programs, such as those
in the {\tt examples} directory.
%Submissions may
%be sent by email to {\tt cs164@cory.eecs.berkeley.edu}.

Test programs will be worth up to 3\% of the total grade, but should
also be much easier to write than optimizers.

\section{Grading}

The final curve for the course will be determined {\em before}
including the extra credit.  In other words, if you elect not to do
extra credit work, you will not be at a disadvantage in the final
grading with respect to those who do.

This extra-credit option is open-ended; you can do as much as you
like.  We will award credit for results. For example, a project that
merely attempts, but does not complete, an optimization phase may
receive as little as no extra credit.

\section{Submission}

\subsection{Optimizer Project}

Submit your optimizer by running ``{\tt /usr/class/cs143/bin/submit}'' 
and selecting assignment ``{\tt PAX1}'', which will collect
the same files as for the code generator assignment (PA5).

\begin{minipage}{0.8\linewidth}
\bigskip
\begin{itemize}
  \item[$\Box$]
    Include a write-up of the optimizations you performed in the \U{README}.

  \item[$\Box$]
    Make sure all your code for the optimizer and code generator is in
    \begin{itemize}
    \item \U{cool-tree.h}, \U{cgen.h},
          \U{cgen.cc}, \U{cgen\_supp.cc}, and \U{emit.h} for the
          C++ version; or
    \item \U{cool-tree.java}, \U{CgenClassTable.java},
          \U{CgenNode.java}, \U{CgenSupport.java}, \U{BoolConst.java},
          \U{IntSymbol.java}, \U{StringSymbol.java},
          \U{TreeConstants.java}, and additional \U{.java} files you
          may have added for the Java version.
    \end{itemize}

  \item[$\Box$]
    Be sure to answer `yes' to the submission prompt for files that contain
    your code.
\end{itemize}
\end{minipage}


\subsection{Test Program Project}

Submit your test program by running ``{\tt /usr/class/cs143/bin/submit}'' 
and selecting assignment ``{\tt PAX2}'', which will collect
all \U{.cl} files.  Your submission will be treated as one test
program; in other words, all \U{.cl} files will be compiled together
into one program.  Please include comments in your source file that
describe your test program.

\end{document}

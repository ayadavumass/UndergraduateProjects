\documentclass[11pt]{article}

\usepackage{handout}

% Side margins:
% Actual margin is 1 in + this number
\oddsidemargin -0.25in
\evensidemargin -0.25in

% Text width:
\textwidth 6.9in

% Top margin:
% Actual margin is 1.5 in + this number
\topmargin -.3in

% Text height:
\textheight 8.7in

% Handout Numbers

\newcommand{\PAOneNum}{1}
\newcommand{\WAOneNum}{2}
\newcommand{\PATwoNum}{3}
\newcommand{\PAThreeNum}{4}
\newcommand{\WATwoNum}{5}
\newcommand{\WATwoSolNum}{6}
\newcommand{\WAOneSolNum}{7}
\newcommand{\WAThreeNum}{8}
\newcommand{\WAFourNum}{9}
\newcommand{\WAThreeSolNum}{10}
\newcommand{\PAFourNum}{11}
\newcommand{\WAFourSolNum}{12}
\newcommand{\WAFiveNum}{13}
\newcommand{\WAFiveSolNum}{14}
\newcommand{\WASixNum}{15}
\newcommand{\WASixSolNum}{16}
\newcommand{\PAFiveNum}{17}
\newcommand{\WASevenNum}{18}
\newcommand{\WASevenSolNum}{19}
\newcommand{\PAExtraNum}{20}
\newcommand{\WAEightNum}{21}
\newcommand{\WAEightSolNum}{22}
\newcommand{\PASixNum}{23}
\newcommand{\WANineNum}{24}
\newcommand{\WANineSolNum}{25}
\newcommand{\WATenNum}{26}
\newcommand{\WATenSolNum}{27}
\newcommand{\WAElevenNum}{28}
\newcommand{\WAElevenSolNum}{29}


\begin{document}

\handout{2}{1}{Instructions for Using Gradiance}

For the written assignments this quarter, we will be making use of
the automated services provided by the company
Gradiance\footnote{\texttt{http://www.gradiance.com}}.  Gradiance
offers a collection of questions about the material covered in a
compilers course.  On at least some of the written assignments, you
will be asked to answer questions on-line at the Gradiance web site.

\section{Setup Instructions}

You can access the Gradiance services at the following URL:

\begin{verbatim}
http://www.gradiance.com/services
\end{verbatim}

\noindent
To use the services, you will first need to create an account.  In
order to simplify the administrative process, when you create your
account, please use your Leland username as the User ID, if possible.
We also request that the name you enter on Gradiance be the same as
the name the registrar lists on the class roster.

After you have created an account, you can sign up for CS143.  You
will need a class token to sign up.  The class token you should enter
is:

\begin{verbatim}
01C78DD8
\end{verbatim}

Once you have signed up for CS143, you can access the CS143 page on
Gradiance by following the link from the home page that you reach when
you log in to Gradiance.  The written assignments will be assigned as
homeworks on Gradiance, and can be accessed by following the Homeworks
link on the CS143 page on Gradiance.

\section{Assignment Format}

The homework assignments on Gradiance consist of multiple-choice
questions.  After you have answered the questions on an assignment,
you submit the assignment, and will be given feedback on your answers.

If you do not answer all of the questions correctly, you will have an
opportunity to repeat the assignment.  The entire assignment must be
completed again, however, as the system always presents all of the
questions, regardless of which questions were answered incorrectly on
the first attempt.  As such, when you repeat an assignment, you must
answer the questions you answered correctly again.

Each time you work on an assignment, the system uses randomization to
generate the list of questions and possible answers to present to you.
For this reason, a particular assignment may look different on each
repetition.

You can submit an assignment an unlimited number of times.  For the
purposes of grading, we will only consider the most recent submission.

Note that because assignments on Gradiance cannot be submitted after
their due dates, the due date listed on Gradiance for an assignment
will be the latest possible date the assignment will be accepted with
the late penalty.  That is, the due date on Gradiance will be three
days after the original due date, which is 5:00 PM on the date listed
on the syllabus.

\end{document}

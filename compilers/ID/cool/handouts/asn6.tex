\documentclass[11pt]{article}
\usepackage{latexsym}
\usepackage{handout}
%
% Copyright (c) 1995-1996 by Alex Aiken.  All rights reserved.
% Permission is granted to modify and distribute this document for
% for non-commercial purposes, so long as this copyright notice is retained
% in all copies.
%
% Side margins:
% Actual margin is 1 in + this number
\oddsidemargin -0.25in
\evensidemargin -0.25in

% Text width:
\textwidth 6.9in

% Top margin:
% Actual margin is 1.5 in + this number
\topmargin -.3in

% Text height:
\textheight 8.7in


% Handout Numbers

\newcommand{\PAOneNum}{1}
\newcommand{\WAOneNum}{2}
\newcommand{\PATwoNum}{3}
\newcommand{\PAThreeNum}{4}
\newcommand{\WATwoNum}{5}
\newcommand{\WATwoSolNum}{6}
\newcommand{\WAOneSolNum}{7}
\newcommand{\WAThreeNum}{8}
\newcommand{\WAFourNum}{9}
\newcommand{\WAThreeSolNum}{10}
\newcommand{\PAFourNum}{11}
\newcommand{\WAFourSolNum}{12}
\newcommand{\WAFiveNum}{13}
\newcommand{\WAFiveSolNum}{14}
\newcommand{\WASixNum}{15}
\newcommand{\WASixSolNum}{16}
\newcommand{\PAFiveNum}{17}
\newcommand{\WASevenNum}{18}
\newcommand{\WASevenSolNum}{19}
\newcommand{\PAExtraNum}{20}
\newcommand{\WAEightNum}{21}
\newcommand{\WAEightSolNum}{22}
\newcommand{\PASixNum}{23}
\newcommand{\WANineNum}{24}
\newcommand{\WANineSolNum}{25}
\newcommand{\WATenNum}{26}
\newcommand{\WATenSolNum}{27}
\newcommand{\WAElevenNum}{28}
\newcommand{\WAElevenSolNum}{29}


\newcommand{\tcrule}[3]{\frac{#1}{#2}\eqno\mbox{#3}}
\newcommand{\attr}[3]{#1:#2\leftarrow#3}
\newcommand{\classmap}[2]{class(#1) = (#2)}
\newcommand{\ossimple}[6]{#1,#2,#3\vdash #4 : #5,#6}
\newcommand{\osrule}[8]{\frac{#7}{\ossimple{#1}{#2}{#3}{#4}{#5}{#6}}\eqno
\mbox{#8}}
\def\U#1{{\sf{}#1}}
\def\S#1{{\tt{}#1}} % NB: we often use \verb+...+ for this also
\def\C#1{{\bf{}#1}}
\newcommand{\dq}{\mbox{\tt "}}

%%%%%%%%%%%%%%%%%%% Additional Macros

\usepackage{xspace}

\newcommand{\MVar}[1]{\ensuremath{\mathit{#1}}\xspace}
\newcommand{\expr}{\MVar{expr}}

\newcommand{\Throw}{\C{throw}\xspace}
\newcommand{\Try}{\C{try}\xspace}
\newcommand{\Catch}{\C{catch}\xspace}
\newcommand{\Finally}{\C{finally}\xspace}

\newcommand{\Arrow}{\mathrel{\mathbf{=>}}\xspace}
\newcommand{\ThrowE}[1]{\ensuremath{\Throw\;{#1}}\xspace}
\newcommand{\TryCatch}[3]{\ensuremath{\Try\;{#1}\;\Catch\;{#2}\Arrow{#3}\;}\xspace}
\newcommand{\TryFinally}[2]{\ensuremath{\Try\;{#1}\;\Finally\;{#2}\;}\xspace}

\newcommand{\Bnfdef}{\mathrel{::=}}
\newcommand{\Bnfalt}{\mid}

\def\@add#1#2{
 \newcount\tmp
 \tmp=#1
 \advance\tmp by #2
 \number\tmp
}
\newcommand{\List}[3]{\ensuremath{
  {#1}_{#2},{#1}_{\@add{#2}{1}},\ldots,{#1}_{#3}
}}
\newcommand{\Coolaid}{\U{Coolaid}}

\begin{document}

\handout{\PASixNum}{6}{Programming Assignment VI \\
Due Tuesday, May 4, 2004 at 11:59 PM}


\section{Introduction}

In this assignment, you will extend \emph{your} code generator for
Cool from Programming Assignment V to handle exceptions.  The
appropriate mechanisms for exceptions have already been added to the
reference lexer, parser, and semantic analyzer for you.

This assignment requires much less code than previous projects; we
have implemented the code generation for exceptions in \U{coolc} in
under 50 lines of C++ code.  However, getting the code correct is
challenging and requires careful thought before implementation.

You may work a group of one or two people for this assignment. The
\U{submit} program will ask you to specify group members when you turn
in your assignment

\section{Files and Directories}

We suggest that you work using the same CVS repository and working
directory as for PA5.  From your working directory, type the following
command to get a copy of the additional files for PA6 (for both C++
and Java):
\begin{verbatim}
  gmake -f ~cs164/assignments/PA6/Makefile source
\end{verbatim}
The additional files for PA6 are as follows:
\begin{itemize}
\item \U{example-PA6.cl} \\
This file should contain a test program of your own design.  Test as
many uses of exceptions as you can.

\item \U{README-PA6}\\
This file will contain the write-up for your assignment. 
It is critical that you explain design decisions, how your
code is structured, and why you believe your design is a good one
(i.e., why it leads to a correct and robust program).  It is part of
the assignment to explain things in text as well as to comment your
code.  Also, please follow the directions in this file to update 
your project files from PA5 with the \U{*.PA6update} files.

\item \U{finally\_exception.cl} \\
This is an example program that uses \Try-\Finally.

\item \U{finally\_exception.output} \\
This file gives the output of running \U{spim} on code generated from
\U{finally\_exception.cl} by \U{coolc}.  You should consult below for a
complete description of the semantics of \Try-\Finally.
\end{itemize}

\section{Exceptions in Cool}

Exceptions are a general mechanism for dealing with error conditions
that arise in computation. It has many advantages over other methods
for handling errors; for example, the programmer does not need to
specify special return values to denote error, and exceptions can be
propagated to the function caller in case the caller has more information
about how to deal with the exception.

\subsection{Syntax}

We extend Cool with exceptions by adding the follow new constructs:
\[ \begin{array}{rrl}
  \expr & \Bnfdef & \ldots \\
        & \Bnfalt & \ThrowE{\expr} \\
        & \Bnfalt & \TryCatch{\expr}{\MVar{ID}}{\expr} \\
        & \Bnfalt & \TryFinally{\expr}{\expr} \\

\end{array} \]
where \Throw, \Try, \Catch, and \Finally are four new keywords.  In
each of the new constructs, the last $\expr$ extends as far as
possible (i.e., encompasses as many tokens as possible).

In Cool, the programmer can throw an exception where the value of the
exception is the result of any Cool expression.  The expression
``\ThrowE{\expr}'' returns the value of $\expr$ to the closest
dynamically enclosing \Catch block.  The expression
``\TryCatch{\expr_1}{\MVar{ID}}{\expr_2}'' evaluates expression
$\expr_1$, and if an exception is thrown in $\expr_1$, then $\expr_2$
is executed with $\MVar{ID}$ bound to the value of the thrown exception.
If no exception is thrown in $\expr_1$, then $\expr_2$ is not executed.
An uncaught exception will be caught by the Cool runtime, in which
case will abort the program.

Note that the form of \Try-\Catch you will implement is slightly
different than in lecture.  Here, the \Catch expression will
handle exceptions of any type, rather than only handling exceptions
with dynamic type that is a subtype of some given type.

The expression ``\TryFinally{\expr_1}{\expr_2}'' evaluates $\expr_1$
and then evaluates $\expr_2$ regardless of whether $\expr_1$ throws an
exception.  If $\expr_1$ throws an exception and $\expr_2$ evaluates
normally, then the exception must continue to propagate after
$\expr_2$ is executed.  However, if an exception is thrown in
$\expr_2$ (regardless of the evaluation of $\expr_1$), only the
exception from $\expr_2$ will continue to propagate.  Finally, if both
$\expr_1$ and $\expr_2$ evaluate normally, then the value for the
entire expression is the value of $\expr_2$ (i.e., it behaves exactly
like ``$\{ \expr_1; \expr_2; \}$'').

\subsection{Static Semantics}

\newcommand{\Ty}[5]{{#1},{#2},{#3} \vdash {#4} : {#5}}

We now extend our typing rules to accommodate these new
constructs.  Recall from Written Assignment 5 the following rules for
\Throw and \Try-\Catch.
$$
\tcrule{ \Ty O M C e {T_1} }%
       { \Ty O M C {\ThrowE e} {T_2} }%
       {[Throw]}
$$
$$
\tcrule{ \begin{array}{l}
           \Ty O M C {e_1} {T_1} \\
           \Ty {O[\textrm{Object}/x]} M C {e_2} {T_2} \\
         \end{array}
       }%
       { \Ty O M C {\TryCatch{e_1}{x}{e_2}} {T_1 \sqcup T_2} }%
       {[Try-Catch]}
$$
A \Throw causes the computation to make a non-local ``jump'', so
the expression ``\ThrowE e'' never ``returns'' to the context in which it
appears syntactically, so can be assigned any type.
We must, however, check that $e$ is well-typed.

The exception caught in a \Try-\Catch block can be of any type, so we
conservatively assume that $x$ has type Object.  Evaluation of $e_1$
can either return normally, in which case the value of the whole
\Try-\Catch expression is the value of $e_1$, or exceptionally, in
which case the value of the whole expression is the value of $e_2$.
Thus, the type of the \Try-\Catch construct is the least upper bound
of $T_1$ and $T_2$.

Now, consider the typing rule for \Try-\Finally.
$$
\tcrule{ \begin{array}{l}
           \Ty O M C {e_1} {T_1} \\
           \Ty O M C {e_2} {T_2} \\
         \end{array}
       }%
       { \Ty O M C {\TryFinally{e_1}{e_2}} {T_2} }%
       {[Try-Finally]}
$$
Since $e_2$ is always evaluated after evaluating $e_1$, the type of
``\TryFinally{e_1}{e_2}'' is the type of $e_2$.


\subsection{Dynamic Semantics}

\newcommand{\Norm}[1]{\ensuremath{\MVar{Norm}(#1)}\xspace}
\newcommand{\Exc}[1]{\ensuremath{\MVar{Exc}(#1)}\xspace}

\newcommand{\Int}[1]{\ensuremath{\MVar{Int}(#1)}\xspace}

\newcommand{\Eval}[6]{\ensuremath{{#1},{#2},{#3} \vdash {#4} : {#5}, {#6}}\xspace}

\newcommand{\so}{\MVar{so}}
\newcommand{\op}{\MVar{op}}

\newcommand{\Id}{\MVar{Id}}

In this section, we extend the operational semantics of Cool to
accommodate the additional syntactic forms for exceptions.  To do this,
we need to define a notion that distinguishes between a normal return
and an exceptional return.  Thus, we define a generalized value $g$ in
terms of conventional Cool values $v$ as follows:
\[ \begin{array}{crl@{\extracolsep{4ex}}l}
  g & \Bnfdef & \Norm{v} & \mbox{a normal return with value $v$} \\
    & \Bnfalt & \Exc{v} & \mbox{an exceptional return with value $v$}
\end{array} \]
Recall that every Cool value is an object, written
\[ \begin{array}{crl@{\extracolsep{4ex}}p{20em}}
  v & \Bnfdef & X(a_1 = l_1, a_2 = l_2, \ldots, a_n = l_n) &
    an object of dynamic type $X$ with attributes
    $\List a 1 n$ at locations $\List l 1 n$ \\
\end{array} \]
We now redefine our evaluation judgment in the following manner to
incorporate these generalized values:
\[
  \Eval \so S E  e g {S'}
\]
This new judgment is read as follows: ``in the context where \MVar{self}
is the object \so, the
store is $S$, and the environment is $E$, the expression $e$ evaluates
to generalized value $g$ and yields the new store $S'$''.

For every rule we had before, we modify the rule to accommodate both
normal return and exceptional return.  For example,
the evaluation rule for arithmetic under normal execution is as follows:
$$
\tcrule{\begin{array}{l}
    \Eval \so {S_1} E   {e_1} {\Norm{\Int{i_1}}} {S_2} \\
    \Eval \so {S_2} E   {e_2} {\Norm{\Int{i_2}}} {S_3} \\
    \op \in \{*,+,-,/\} \\
    v = \Int{i_1\;\op\;i_2}
    \end{array}
}{\Eval \so {S_1} E  {e_1\;\op\;e_2} {\Norm{v}} {S_3}}
 {[Arith]}
$$
However, in the case that $e_1$ or $e_2$ returns
exceptionally, we need to propagate the exception as
follows:
$$      
\tcrule{
    \begin{array}{l}
    \Eval \so {S_1} E   {e_1} {\Exc{v_1}} {S_2} \\
    \op \in \{*,+,-,/\}
    \end{array}
}{
    \Eval \so {S_1} E   {e_1\;\op\;e_2} {\Exc{v_1}} {S_2}
}{[Arith-Exc1]}
$$
$$
\tcrule{
    \begin{array}{l}
    \Eval \so {S_1} E   {e_1} {\Norm{v_1}} {S_2} \\
    \Eval \so {S_2} E   {e_2} {\Exc{v_2}} {S_3} \\
    \op \in \{*,+,-,/\}
    \end{array}
}{
    \Eval \so {S_1} E   {e_1\;\op\;e_2} {\Exc{v_2}} {S_3} \\
}{[Arith-Exc2]}
$$
Note that in [Arith-Exc1], the evaluation of $e_2$ is
aborted.  Also, notice that the store is threaded through all expressions,
which means that all side-effects are propagated.

For \Throw, there are two cases to consider depending on whether the
expression returns normally or exceptionally.  If the expression
returns normally with object $v$, then $v$ is value of the thrown
exception.
$$
\tcrule{
    \Eval \so {S_1} E  e {\Norm v} {S_2}
}{
    \Eval \so {S_1} E  {\ThrowE e} {\Exc v} {S_2}
}{[Throw]}
$$
Otherwise, if the expression returns exceptionally, that
exception should still be propagated.
$$
\tcrule{
    \Eval \so {S_1} E  e {\Exc v} {S_2}
}{
    \Eval \so {S_1} E  {\ThrowE e} {\Exc v} {S_2}
}{[Throw-Exc]}
$$
Note that an exceptional value is always returned.

For \Try-\Catch, the operational semantics are as follows.
$$
\tcrule{
    \Eval \so {S_1} E   {e_1}  {\Norm{v_1}} {S_2}
}{
    \Eval \so {S_1} E   {\TryCatch{e_1}{\Id}{e_2}}  {\Norm{v_1}} {S_2}
}{[Try-Catch]}
$$
If $e_1$ return normally, then the \Try-\Catch simply
returns that value.  Otherwise, if $e_1$ returns exceptionally, then
the \Catch expression is evaluated.
$$
\tcrule{
    \begin{array}{l}
    \Eval \so {S_1} E   {e_1}  {\Exc{v_1}} {S_2} \\
    l_1 = \MVar{newloc}(S_2) \\
    S_3 = S_2[v_1/l_1] \\
    E' = E[l_1/\Id] \\
    \Eval \so {S_3} {E'}   {e_2}  {g_2} {S_4}
    \end{array}
}{
    \Eval \so {S_1} E   {\TryCatch{e_1}{\Id}{e_2}}  {g_2} {S_4}
}{[Try-Catch-Exc]}
$$
Note that $g_2$ can be either a normal or an exceptional
value.

The evaluation rule(s) for \Try-\Finally is/are left as an exercise
that you will answer in \U{README-PA6}.


\section{Code Generation}

\newcommand{\FP}{\ensuremath{\mathtt{\$fp}}\xspace}
\newcommand{\SP}{\ensuremath{\mathtt{\$sp}}\xspace}
\newcommand{\XP}{\ensuremath{\mathtt{\$xp}}\xspace}
\newcommand{\TNine}{\ensuremath{\mathtt{\$t9}}\xspace}
\newcommand{\AZero}{\ensuremath{\mathtt{\$a0}}\xspace}

You will implement the long jump scheme to implement exceptions. In
this approach, for every \Try-\Catch, some state information is pushed
onto the stack (called an \emph{exception frame}) so that it can be
restored when execution passes to the \Catch block to handle an
exception.  A special \emph{exception register} points to the most
recent exception frame, with each exception frame containing a pointer
to the previous exception frame in a linked-list fashion.  For
convenience, we let \XP denote the exception register (note that this
is not an actual register name on the MIPS architecture).  To be
compatible with the runtime system and \Coolaid, you should use \TNine
as the exception register for your code generator.  On a \Throw,
the value of the exception is put in \AZero, the
exception register is read, and control is passed to the \Catch block
specified in the exception frame.  Note that the exception is
returned to the \Catch block in \AZero.  The \Catch block is responsible for
first restoring the state from the exception frame.  The exception
frame is then popped by restoring the exception pointer to point to
the previous exception frame.

\subsection{Exception Frame}

The exception frame must store (either implicitly or explicitly) state
information in the evaluation of the \Try block before the occurrence
of the exception to evaluate the \Catch block to handle the exception.
To be compatible with the top-level exception handler in the Cool
runtime system, the layout of the exception frame should be as
follows:
\begin{center}
\begin{tabular}{l|c|l}
\cline{2-2}
  $\XP \longrightarrow$ & catch label & low addresses \\ \cline{2-2}
  & \\
  & \vdots \\
  & & high addresses \\
%  & saved self object \\ \cline{2-2}
%  & saved stack pointer (\SP) \\ \cline{2-2}
%  & saved frame pointer (\FP) \\ \cline{2-2}
%  & next exception frame pointer & high addresses \\
\cline{2-2}
\end{tabular}
\end{center}
The first word of the exception frame contains the address of the
label of the closest dynamically enclosing \Catch block.  The catch
label will indicate where to jump to on a \Throw.  Thus, the final two
instructions of the code generation for a \Throw is something like
\begin{verbatim}
        lw    $t1 0($t9)
        jr    $t1
\end{verbatim}

It is up to you how to layout the rest of the exception frame.  In
general, the exception frame saves the state information that needs
restored to evaluate the \Catch block.  Minimally, you will need to be
able to restore the stack pointer, the frame pointer, and a pointer to
the previous exception frame.

%In addition to the stack pointer and frame pointer, we need to save
%the self object to fully restore to context of evaluation for Cool
%programs.
%because we could be returned exceptionally from a
%different object, in which case the self object needs to be restored
%for correct lookup into method dispatch table and formal variables.

%The frame pointer will allow us to reset the frame to the same
%location as when we entered the try/catch block. The old exception
%register is restored since we just 'used' an exception frame, and need
%to restore the previous exception frame.

%Specifically, on every try/catch statement, generate a new label at
%the beginning of the catch expression, and one at the very end of the
%try/catch statement. Now, decrement the stack and save the the above
%information onto the stack. Make sure to set the exception pointer
%point to the value of the stack pointer (do we can restore the stack
%pointer later). Now, generate the code for the expression inside the
%try, and generate a jump to the end of the try/catch statement (so
%execution can skip the code inside the catch statement in case
%execution is normal). Next, generate code to bind the identifier in
%the catch statement to the accumulator register (like in a let
%statement), and generate the code for the statement inside the catch.
%
%For every throw statement, you will essentially restore the
%information that was saved in the try/catch. First, generate the code
%for expression inside the throw, saving the result into the
%accumulator register. Then, set the stack pointer to be the exception
%pointer, as this was the value of the stack pointer when we entered
%the exception. Next, load all the fields of the exception frame into
%their appropriate registers. Restore the catch label into a convenient
%register (such as \$t1). Finally, increment the stack to 'remove' the
%exception frame, and jump to the register holding to catch label in
%execute the catch code.
%
%Make sure to adhere to the layout we have provided, as this is the
%same layout we used to ensure that uncaught exceptions are correctly
%handled by the trap handler.


\subsection{Runtime Support}

In order to handle uncaught exceptions, we have installed a top-level
exception frame in the trap handler before your Cool program is
invoked.  This exception frame is laid out according to the layout
described in the previous section so that the code generation of
\Throw following this layout will jump to a special ``\Catch'' block
provided by the runtime system.  This runtime \Catch block prints the
message ``Uncaught Exception'' and aborts the Cool program.  You can
look in the \S{trap.handler} to see how this implemented. Look for the
labels \S{\_\_exception} and \S{\_\_exception\_handler}.

\subsection{Try-Finally}

Note that code generation for \Try-\Finally is not enabled in
\U{coolc}.  You should first think carefully about the expected
behavior of a \Try-\Finally expression and an implementation strategy
before coding.


\section{Testing and Debugging}

As with Programming Assignment V, you will need a scanner, parser, and
semantic analyzer that handles exceptions to test your code
generator. You may extend your own components or the components from
\U{coolc}.  By default, the \U{coolc} components are used.  To change
this behavior, replace the \U{lexer}, \U{parser}, and/or \U{semant} executable
(which are symbolic links in your project directory) with your own
scanner/parser.  Even if you use your own components, it is wise to
test your code generator with the \U{coolc} scanner, parser, and
semantic analyzer at least once because we will grade your project
using \U{coolc}'s version of the other phases.

We have included a number of test programs using the exception
constructs.  They can be found in \U{\~{}cs164/examples/}.  All
exception programs are named \U{*\_exception.cl}.  \Coolaid{} is able
to verify programs with exceptions with the exception frame layout and
code generation for \Throw given above.  Try running your compiler on
these examples and verifying the result in \Coolaid{}.  If you use
\Coolaid{}, please help us by providing feedback about it in the
\U{README-PA6} (as in Programming Assignment V).  You may also use
\U{spim}/\U{xspim} to running code generated from your compiler on
these examples.  These examples are by no means comprehensive. You
should also develop your own test cases.

\section{Grading}

To minimize the effect of incorrect solutions for Programming
Assignment V, the test cases that we will use to grade this project
will not require code generation for \S{new SELF\_TYPE} nor for static
dispatch.


\newpage
\section{Final Submission}

Make sure to complete the following items before submitting to avoid
any penalties.  Then, submit your project by typing ``\S{submit
PA6}''.

\begin{minipage}{0.8\linewidth}
\bigskip
\begin{itemize}
  \item[$\Box$]
    Include your write-up in \U{README-PA6}.  Please include your
    feedback about \Coolaid{} in \U{README-PA6}.

  \item[$\Box$]
    Include your test cases that test your code generator with
    exceptions in \U{example-PA6.cl}.

  \item[$\Box$]
    Make sure all your code for the code generator with exceptions
    is in
    \begin{itemize}
    \item \U{cool-tree.h}, \U{cgen.h},
          \U{cgen.cc}, \U{cgen\_supp.cc}, and \U{emit.h} for the
          C++ version; or
    \item \U{cool-tree.java}, \U{CgenClassTable.java},
          \U{CgenNode.java}, \U{CgenSupport.java}, \U{BoolConst.java},
          \U{IntSymbol.java}, \U{StringSymbol.java},
          \U{TreeConstants.java}, and additional \U{.java} files you
          may have added for the Java version.
    \end{itemize}

  \item[$\Box$]
    Be sure to answer `yes' to the submission prompt for files that contain
    your code.
\end{itemize}
\end{minipage}

\end{document}

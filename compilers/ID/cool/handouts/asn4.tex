\documentclass[11pt]{article}
\usepackage{handout}
%
% Copyright (c) 1995-1996 by Alex Aiken.  All rights reserved.
% Permission is granted to modify and distribute this document for
% for non-commercial purposes, so long as this copyright notice is retained
% in all copies.
%
\input{mymargins}
\input{macros}
% Handout Numbers

\newcommand{\PAOneNum}{1}
\newcommand{\WAOneNum}{2}
\newcommand{\PATwoNum}{3}
\newcommand{\PAThreeNum}{4}
\newcommand{\WATwoNum}{5}
\newcommand{\WATwoSolNum}{6}
\newcommand{\WAOneSolNum}{7}
\newcommand{\WAThreeNum}{8}
\newcommand{\WAFourNum}{9}
\newcommand{\WAThreeSolNum}{10}
\newcommand{\PAFourNum}{11}
\newcommand{\WAFourSolNum}{12}
\newcommand{\WAFiveNum}{13}
\newcommand{\WAFiveSolNum}{14}
\newcommand{\WASixNum}{15}
\newcommand{\WASixSolNum}{16}
\newcommand{\PAFiveNum}{17}
\newcommand{\WASevenNum}{18}
\newcommand{\WASevenSolNum}{19}
\newcommand{\PAExtraNum}{20}
\newcommand{\WAEightNum}{21}
\newcommand{\WAEightSolNum}{22}
\newcommand{\PASixNum}{23}
\newcommand{\WANineNum}{24}
\newcommand{\WANineSolNum}{25}
\newcommand{\WATenNum}{26}
\newcommand{\WATenSolNum}{27}
\newcommand{\WAElevenNum}{28}
\newcommand{\WAElevenSolNum}{29}


\usepackage{latexsym}

\begin{document}

\handout{6}{6}{Programming Assignment III \\
Due Thursday, November 6, 2008 at 11:59 PM }

% Three macros for defining:
%       Unix elements: filenames and program (sans serif)
%       Sather164 elements: literal tokens (typewriter)
%       C elements: function and variable names (boldface)
%
\def\U#1{{\sf{}#1}}
\def\S#1{{\tt{}#1}} % NB: we often use \verb+...+ for this also
\def\C#1{{\bf{}#1}}

\bigskip
\section{Introduction} 

In this assignment, you will implement the static semantics of Cool.  You
will use the abstract syntax trees (AST) built by the parser to check
that a program conforms to the Cool specification. Your
static semantic component should reject erroneous programs; for correct
programs, it must gather certain information for use by the code
generator.  The output of the semantic analyzer will be an annotated
AST for use by the code generator.

This assignment has much more room for design decisions than previous
assignments. Your program is correct if it checks programs against the
specification. There is no one ``right'' way to do the assignment, but there
are wrong ways. There are a number of standard practices that we think
make life easier, and we will try to convey them to you. However, what
you do is largely up to you.  Whatever you decide to do, be prepared to
justify and explain your solution.

You will need to refer to the typing rules, identifier scoping rules,
and other restrictions of Cool as defined in the Cool Reference
Manual.  You will also need to add methods and data members to the AST
class definitions for this phase.  The functions the tree package
provides are documented in the {\em Tour of Cool Support Code}.

There is a lot of information in this handout, and you need to know most
of it to write a working semantic analyzer.  {\em Please read the
handout thoroughly.}
At a high level, your semantic
checker will have to perform the following major tasks:
\begin{enumerate}
\item Look at all classes and build an inheritance graph.
\item Check that the graph is well-formed.
\item For each class
\begin{enumerate}
\item Traverse the AST, gathering all visible declarations in a symbol table.
\item Check each expression for type correctness.
\item Annotate the AST with types.
\end{enumerate}
\end{enumerate}
This list of tasks is not exhaustive; it is up to you to faithfully
implement the specification in the manual.


You may work in a group of one or two people.

%You must work in a group for this assignment (where a group consists of
%one or two people).  The \U{submit} program will ask you to specify
%group members when you turn in your assignment.

\section{Files and Directories}

%We strongly suggest you use CVS for this assignment.  This assignment
%is larger than the previous ones, so version control may be especially
%helpful.
%
%To use CVS on the instructional machines, you need to have a cvs group
%id (see the web page).  You will need to run ``{\tt cvsgroup cs164-gNN}'',
%where NN is your group number, each time you login -- so putting this
%command in your .cshrc file is a good idea.
%
%To create a CVS repository for the C++ version of the project, run:
%
%\begin{verbatim}
%~cs164/etc/make-cvs PA4 PA4
%\end{verbatim}
%
%\noindent For Java, type:
%
%\begin{verbatim}
%~cs164/etc/make-cvs PA4J PA4
%\end{verbatim}
%
%\noindent
%(notice the extra ``J'').  The first argument to make-cvs is the
%subdirectory of {\tt \~{}cs164/assignments} that you want to use, while the
%second is the name of the CVS repository you want to create. Now you
%(and your partner, if you have one) can run ``{\tt cvs checkout PA4}'',
%which will create a subdirectory PA4 in the current working
%directory.  Now cd to the PA4 directory.  If you are using C++, run this
%command: 

To get started, create a directory where you want to do the assignment
and execute one of the following commands {\em in that directory}.  For
the C++ version of the assignment, you should type

\begin{verbatim}
gmake -f /usr/class/cs143/assignments/PA4/Makefile
\end{verbatim}

\noindent For Java, type:

\begin{verbatim}
gmake -f /usr/class/cs143/assignments/PA4J/Makefile
\end{verbatim}

\noindent
(notice the ``J'' in the path name).

%This will create several
%symbolic links in the currect directory.  You should not edit the
%files pointed to by these links.  In fact, if you make and modify
%private copies of these files, you may find it impossible to complete
%the assignment.
%See the instructions in the \U{README} file.  

As usual, there are several files used in the assignment that are
symbolically linked to your directory or are included from \U{\
/usr/class/cs143/include/PA4}.  Do {\bf not} modify these files.  Almost all of
these files have have been described in previous assignments. See the
instructions in the \U{README} file.

%{\bf Important:} All software supplied with this assignment is
%supported on Solaris SPARC, Solaris x86, and Linux x86.  Remember to
%run \U{gmake clean} if you switch architectures.

We now describe the most important files for each
version of the project.

\subsection{C++ Version}

This is a list of the files that you may want to modify.

\begin{itemize}

\item \U{cool-tree.h} \\
This file is where user-defined extensions to the abstract syntax
tree nodes are placed.
%You can modify either file, but you will
%probably find it easiest to modify \U{cool-tree.h}.
You will likely
need to add additional declarations, but do {\bf not} modify the
existing declarations.


\item \U{semant.cc} \\
This is the main file for your implementation of the semantic analysis
phase.  It contains some symbols predefined for your convenience and a
start to a \C{ClassTable} implementation for representing the
inheritance graph.  You may choose to use or ignore these.

The semantic analyzer is invoked by calling method \C{semant()} of
class \C{program\_class}.  The class declaration for
\C{program\_class} is in \U{cool-tree.h}.  Any method declarations you
add to \U{cool-tree.h} should be
implemented in this file.


\item \U{semant.h} \\
This file is the header file for \U{semant.cc}.  You add any
additional declarations you need (not in \U{cool-tree.h})
here.

\item \U{good.cl} and \U{bad.cl} \\
These files test a few semantic features.  You should add tests to
ensure that \U{good.cl} exercises as many legal semantic combinations as
possible and that \U{bad.cl} exercises as many kinds of semantic errors
as possible.  It is not possible to exercise all possible combinations
in one file; you are only responsible for achieving reasonable coverage.
Explain your tests in these files and put any overall comments in the
\U{README} file.

\item \U{README} \\
This file will contain the write-up for your assignment.  For this
assignment, it is critical that you explain design decisions, how your
code is structured, and why you believe that the design is a good one
(i.e., why it leads to a correct and robust program).  It is part of the
assignment to explain things in text, as well as to comment your code.
Inadequate \U{README} files will be penalized more heavily in this
assignment, as the \U{README} is the major guideline we have to
understanding your code.

\end{itemize}


\subsection{Java Version}

This is a list of the files that you may want to modify.

\begin{itemize}
\item \U{cool-tree.java} \\
This file contains the definitions for the AST nodes and is the
main file for your implementation of the semantic analysis phase.
You will need to
add the code for your semantic analysis phase in this file.  The
semantic analyzer is invoked by calling method \C{semant()} of class
\C{program}. Do {\bf not} modify the existing declarations.

\item \U{ClassTable.java} \\
This class is a placeholder for some useful methods (including
error reporting and initialization of basic classes).  You may wish to
enhance it for use in your analyzer.

\item \U{TreeConstants.java} \\
This file defines some useful symbol constants.

\item \U{good.cl} and \U{bad.cl} \\
These files test a few semantic features.  You should add tests to
ensure that \U{good.cl} exercises as many legal semantic combinations as
possible and that \U{bad.cl} exercises as many kinds of semantic errors
as possible.  It is not possible to exercise all possible combinations
in one file; you are only responsible for achieving reasonable coverage.
Explain your tests in these files and put any overall comments in the
\U{README} file.

\item \U{README} \\
This file will contain the write-up for your assignment.  For this
assignment it is critical that you explain design decisions, how your
code is structured, and why you believe that the design is a good one
(i.e., why it leads to a correct and robust program).  It is part of the
assignment to explain things in text as well as to comment your code.
Inadequate \U{README} files will be penalized more heavily in this
assignment, as the \U{README} is the major guideline we have to
understanding your code.


\end{itemize}

\section{Tree Traversal}

As a result of assignment 3, your parser builds abstract syntax trees.
The method \C{dump\_with\_types}, defined on most AST nodes, illustrates
how to traverse the AST and gather information from it. This algorithmic
style---a recursive traversal of a complex tree structure---is very
important, because it is a very natural way to structure many
computations on ASTs.

Your programming task for this assignment is to (1) traverse the tree, (2)
manage various pieces of information that you glean from the tree, and
(3) use that information to enforce the semantics of Cool.  One traversal
of the AST is called a ``pass''.  You will probably need to make at
least two passes over the AST to check everything.

You will most likely need to attach customized information to the AST
nodes.  To do so, you may edit \U{cool-tree.h} (C++) or
\U{cool-tree.java} (Java) directly.  In the C++ version, the method
implementations you wish to add should go into \U{semant.cc}.

\section{Inheritance}

Inheritance relationships specify a directed graph of class
dependencies.  A typical requirement of most languages with inheritance
is that the inheritance graph be acyclic.  It is up to your semantic
checker to enforce this requirement. One fairly easy way to do this is
to construct a representation of the type graph and then check for
cycles.

In addition, Cool has restrictions on inheriting from the basic classes
(see the manual).  It is also an error if class \U{A} inherits from
class \U{B} but class \U{B} is not defined.

The project skeleton includes appropriate definitions of all the basic
classes. You will need to incorporate these classes into the inheritance
hierarchy.

We suggest that you divide your semantic analysis phase into two smaller
components.  First, check that the inheritance graph is well-defined,
meaning that all the restrictions on inheritance are satisfied.  If the
inheritance graph is not well-defined, it is acceptable to abort
compilation (after printing appropriate error messages, of course!).
Second, check all the other semantic conditions.  It is much easier to
implement this second component if one knows the inheritance graph and
that it is legal.

\section{Naming and Scoping}

A major portion of any semantic checker is the management of names.  The
specific problem is determining which declaration is in effect for each
use of an identifier, especially when names can be reused. For example,
if \C{i} is declared in two \U{let} expressions, one nested within the
other, then wherever \C{i} is referenced the semantics of the language
specify which declaration is in effect.  It is the job of the semantic
checker to keep track of which declaration a name refers to.

As discussed in class, a {\em symbol table} is a convenient data
structure for managing names and scoping.  You may use our
implementation of symbol tables for your project.  Our implementation
provides methods for entering, exiting, and augmenting scopes as needed.
You are also free to implement your own symbol table, of course.

Besides the identifier \C{self}, which is implicitly bound in every
class, there are four ways that an object name can be introduced in
Cool:
\begin{itemize}
\item attribute definitions;
\item formal parameters of methods;
\item let expressions;
\item branches of case statements.
\end{itemize}

In addition to object names, there are also method names and class
names.  It is an error to use any name that has no matching
declaration.  In this case, however, the semantic analyzer should
{\em not} abort compilation after discovering such an error.
Remember that neither classes, methods, nor attributes need be declared
before use.  Think about how this affects your analysis.

\section{Type Checking}

Type checking is another major function of the semantic analyzer.  The
semantic analyzer must check that valid types are declared where
required.  For example, the return types of methods must be declared.
Using this information, the semantic analyzer must also verify that
every expression has a valid type according to the type rules.  The
type rules are discussed in detail in the Cool Reference Manual and
the course lecture notes.

One difficult issue is what to do if an expression doesn't have a valid
type according to the rules.  First, an error message should be printed
with the line number and a description of what went wrong.  It is
relatively easy to give informative error messages in the semantic
analysis phase, because it is generally obvious what the error is.  We
expect you to give informative error messages.  Second, the semantic
analyzer should attempt to recover and continue.
%A good semantic
%analyzer will avoid cascading errors using any of several standard
%techniques.
We do expect your semantic analyzer to recover, but we do
not expect it to avoid cascading errors.  A simple recovery mechanism is
to assign the type \U{Object} to any expression that cannot otherwise be
given a type (we used this method in \U{coolc}). 

\section{Code Generator Interface}

For the semantic analyzer to work correctly with the rest of the
\U{coolc} compiler, some care must be taken to adhere to the interface
with the code generator.  We have deliberately adopted a very simple,
na\"\i{}ve interface to avoid cramping your creative impulses in semantic
analysis.  However, there is one thing you must do.  For every
expression node, its \U{type} field must be set to the \U{Symbol} naming
the type inferred by your type checker. This \U{Symbol} must be the
result of the \U{add\_string} (C++) or \U{addString} (Java) method of
the \U{idtable}.  The special expression \U{no\_expr} must be assigned
the type \U{No\_type} which is a predefined symbol in the project
skeleton.

\section{Expected Output}

For incorrect programs, the output of semantic analysis is error
messages.  You are expected to recover from all errors except for
ill-formed class hierarchies.  You are also expected to produce complete
and informative errors.  Assuming the inheritance hierarchy is
well-formed, the semantic checker should catch and report all semantic
errors in the program.
%When in doubt, use \C{coolc} as a guide in
%determining what informative error messages should say.
Your error
messages need not be identical to those of \C{coolc}.

We have supplied you with simple error reporting methods
\C{ostream\& ClassTable::semant\_error(Class\_)} (C++)
and
\C{PrintStream ClassTable.semantError(class\_)} (Java).
%There is one routine that takes a filename and the AST node
%where the error occurred, and it returns the error stream after it has
%printed an error header.  The filename should be the file in which the
%error occurs.
This routine takes a \U{Class\_} (C++) or \U{class\_} (Java) node
and returns an output stream that you can use to write error
messages.  Since the parser ensures that \U{Class\_}/\U{class\_} nodes
store the file in which the class was defined (recall that class
definitions cannot be split across files), the line number of the
error message can be obtained from the AST node where the error is
detected and the filename from the enclosing class.

For correct programs, the output is a type-annotated abstract syntax
tree.  You will be graded on whether your semantic phase correctly
annotates ASTs with types and on whether your semantic phase works
correctly with the \C{coolc} code generator.

%You are also expected to program in good, structured style.  You should
%spend some time thinking about the class definitions you will use.

%Grading will be out of 50 points, with 4 points for the README, 4 for
%the test cases in good.cl and bad.cl, 2 for comments, and 40 points
%for correctly typechecking our tests.


\section{Testing the Semantic Analyzer}

You will need a working scanner and parser to test your semantic
analyzer. You may use either your own scanner/parser or the \U{coolc}
scanner/parser.  By default, the \U{coolc} phases are used; to change
that, replace the \U{lexer} and/or \U{parser} executable (which are
symbolic links in your project directory) with your own scanner/parser.
Even if you use your own scanner and/or parser, it is wise to test your
semantic analyzer with the \U{coolc} scanner and parser at least once,
because we will grade your semantic analyzer using \U{coolc}'s scanner
and parser.

You will run your semantic analyzer using \U{mysemant}, a shell script
that ``glues'' together the analyzer with the parser and the scanner.
Note that \U{mysemant} takes a \U{-s} flag for debugging the analyzer;
using this flag merely causes {\tt semant\_debug} (a global variable in
the C++ version and a static field of class {\tt Flags} in the Java
version) to be set.  Adding the actual code to produce useful debugging
information is up to you.  See the project \U{README} for details.

Once you are confident that your semantic analyzer is working, try running
\U{mycoolc} to invoke your analyzer together with other compiler
phases. You should test this compiler on both good and bad inputs to see
if everything is working. Remember, bugs in the semantic analyzer may
manifest themselves in the code generated or only when the compiled
program is executed under \U{spim}.



\section{Remarks}
\label{check}

The semantic analysis phase is by far the largest component of the
compiler so far.  Our solution is approximately 1300 lines of
well-documented C++.  You will find the assignment easier if you take
some time to design the semantic checker prior to coding.  Ask yourself:

\begin{itemize}
\item What requirements do I need to check?
\item When do I need to check a requirement?
\item When is the information needed to check a requirement generated?
\item Where is the information I need to check a requirement?
\end{itemize}

If you can answer these questions for each aspect of Cool, implementing
a solution should be straight-forward.  

\section{Final Submission}	

Make sure to complete the following items before submitting to avoid
any penalties.

\begin{minipage}{0.8\linewidth}
\bigskip
\begin{itemize}
  \item[$\Box$]
    Include your write-up in \U{README}.

  \item[$\Box$]
    Include your test cases that should pass the semantic analyzer in
    \U{good.cl} and your test cases that should cause the semantic
    analyzer to issue an error in \U{bad.cl}.

  \item[$\Box$]
    Make sure all your code for the semantic analyzer is in
    \begin{itemize}
    \item \U{cool-tree.h}, \U{semant.h}, and
          \U{semant.cc} for the C++ version; or
    \item \U{cool-tree.java}, \U{ClassTable.java}, and
          \U{TreeConstants.java}. % , and additional \U{.java} files you
          % may have added for the Java version.
    \end{itemize}

\end{itemize}
\end{minipage}


%\section{An optional tool that may be of use to you}
%Harmonia-mode is back, and this time it supports both Cool {\it and} 
%Java. To remind you, Harmonia-mode is produced by the researchers in 
%Prof. Susan L. Graham's Harmonia project. It is an extension to XEmacs 
%that will analyze your Cool and Java programs as you're editing them, 
%and give you syntax highlighting, proper indentation, and most 
%importantly, will highlight syntax and semantics errors as you type.
%
%Harmonia-mode for Java is new as of this week, and you may try it out 
%on the instructional machines. While we must reiterate that 
%Harmonia-mode is research technology, and therefore may crash on you 
%(though, it has been engineered to recover from any type of crash and 
%preserve your work), we have fixed every bug reported to us during your 
%explorations with Harmonia-mode for Cool. Hopefully, you will all find 
%it much harder to crash this time.
%
%As far as performance, no improvements were made since you last used 
%Harmonia-mode for Cool. Harmonia-mode does prefer to use as much CPU 
%power as it can get, so running it on heavily-loaded instructional 
%servers will probably lead to poor performance. If you can, try to find 
%an unloaded machine to run it.
%
%We encourage those souls who like to live on the bleeding-edge to try 
%it out at http://www.cs.berkeley.edu/\~{}harmonia/harmonia/cs164.  {\em Use of
%this tool is optional,} and will not affect your grade.  If you like 
%it (or don't) tell us in your README.  Feel free to email the Harmonia 
%group directly as well, at harmonia-feedback@sequoia.cs.berkeley.edu.
%
%If you used Harmonia-mode to edit Java for all or part of your project, 
%please take a minute to answer the questions given in the README.

%\begin{enumerate}
%\item Rate Harmonia-mode's usefulness on a scale of 0 to 6, where 0 
%means that it kept you from doing any work, 3 means that it neither 
%helped nor hindered you, and 6 means that you'll never again dare to 
%edit a Java program without it.
%\item What aspects of Harmonia-mode particularly helped you?
%\item What aspects were particularly detrimental?
%\end{enumerate}




\end{document}


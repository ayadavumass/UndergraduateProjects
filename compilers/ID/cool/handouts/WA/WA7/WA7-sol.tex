\documentclass{article}

\usepackage{../handout}
\usepackage{fullpage}
\usepackage{epsfig}
\usepackage{graphicx}

% Handout Numbers

\newcommand{\PAOneNum}{1}
\newcommand{\WAOneNum}{2}
\newcommand{\PATwoNum}{3}
\newcommand{\PAThreeNum}{4}
\newcommand{\WATwoNum}{5}
\newcommand{\WATwoSolNum}{6}
\newcommand{\WAOneSolNum}{7}
\newcommand{\WAThreeNum}{8}
\newcommand{\WAFourNum}{9}
\newcommand{\WAThreeSolNum}{10}
\newcommand{\PAFourNum}{11}
\newcommand{\WAFourSolNum}{12}
\newcommand{\WAFiveNum}{13}
\newcommand{\WAFiveSolNum}{14}
\newcommand{\WASixNum}{15}
\newcommand{\WASixSolNum}{16}
\newcommand{\PAFiveNum}{17}
\newcommand{\WASevenNum}{18}
\newcommand{\WASevenSolNum}{19}
\newcommand{\PAExtraNum}{20}
\newcommand{\WAEightNum}{21}
\newcommand{\WAEightSolNum}{22}
\newcommand{\PASixNum}{23}
\newcommand{\WANineNum}{24}
\newcommand{\WANineSolNum}{25}
\newcommand{\WATenNum}{26}
\newcommand{\WATenSolNum}{27}
\newcommand{\WAElevenNum}{28}
\newcommand{\WAElevenSolNum}{29}


\newcommand{\ttmath}[1]{$\mathtt{#1}$}
\newcommand{\ossimple}[6]{#1,#2,#3\vdash #4 : #5,#6}
\newcommand{\osrule}[8]{\frac{#7}{\ossimple{#1}{#2}{#3}{#4}{#5}{#6}}\eqno
  \mbox{#8}}

\begin{document}

\handout{\WASevenSolNum}{2}{Solutions to Written Assignment 7}

\begin{enumerate}

\item

\begin{enumerate}

\item The following diagram illustrates objects of type {\tt A} and {\tt
B} along with their dispatch tables.

\begin{center}
\includegraphics{objects}
\end{center}

\item Note that we assume the existence of a label called {\tt
dispatch\_error} that handles the case where {\tt \$a0} is null.

\begin{verbatim}
    beq    $a0 $zero dispatch_error      # check for null obj
    lw     $t0 8($a0)                    # load dispatch pointer
    lw     $t0 16($t0)                   # load method2 ptr from table
    jalr   $t0                           # jump to method
\end{verbatim}

\item

If {\tt obj} has dynamic type {\tt B}, then we correctly invoke {\tt
method2} on the object.  All objects have a dispatch pointer at offset 8,
so we correctly fetch the dispatch pointer.  Furthermore, all classes that
inherit from class {\tt A} will have a pointer to the appropriate version
of {\tt method2} at offset 16 of their dispatch table.  In this case, we
will call the version of {\tt method2} supplied by class {\tt A}, since
class {\tt B} did not override it.

Note that it is legal to pass an object of type {\tt B} as the {\tt self}
object, since the layout of {\tt A} is a prefix of the layout of {\tt B}.

\end{enumerate}

\item 

These rules treat arrays as objects with n attributes, all of type $T$.
For convenience, arrays are indexed from $1$ to $n$, rather than $0$ to
$n-1$. 

$$
\osrule{so}{S_1}{E_1}{\mbox{let } a:T_0[e_1] \mbox{ in } e_2}{v_2}{S_{4}}
	{\begin{array}{l}
	T = \left\{ {\begin{array}{ll}
	           X & \mbox{if } T_0 = \mbox{ SELF\_TYPE and } so = X(\dots)\\
		   T_0 & \mbox{otherwise}
                   \end{array}}\right.\\
	\ossimple{so}{S_1}{E_1}{e_1}{\mbox{Int}(n)}{S_2}\\
	l_i = \mbox{newloc}(S_2) \mbox{ for } i = 0 \dots n \mbox{ and each } l_i \mbox{ is distinct }\\
	v_a = \mbox{array}(a_1:l_1, \dots, a_n:l_n)\\
	S_3 = S_2[v_a/l_0, D_T/l_1, \dots, D_T/l_n]\\
	E_2 = E_1[l_0/a]\\
	\ossimple{so}{S_3}{E_2}{e_2}{v_2}{S_4}
	 \end{array}}{[Array-Let]}
$$

$D_T$ is just the default value for objects of type $T$ (e.g., 0 for
\texttt{Int}s and \texttt{void} for most objects).

$$
\osrule{so}{S_1}{E}{a[e_1] \mbox{\texttt{ <- }} e_2}{v_2}{S_{4}}
	{\begin{array}{l}
	\ossimple{so}{S_1}{E}{e_1}{\mbox{Int}(m)}{S_2}\\
	\ossimple{so}{S_2}{E}{e_2}{v_2}{S_3}\\
	E(a) = l_a\\
	S_2(l_a) = v_a\\
	v_a = \mbox{array}(a_1:l_1, \dots, a_n:l_n)\\
	1 \le m \le n\\
	S4 = S3[v_2/l_m]
	 \end{array}}{[Array-Assign]}
$$

$$
\osrule{so}{S_1}{E}{a[e]}{v}{S_2}
	{\begin{array}{l}
	\ossimple{so}{S_1}{E}{e_1}{\mbox{Int}(m)}{S_2}\\
	E(a) = l_a\\
	S_2(l_a) = v_a\\
	v_a = \mbox{array}(a_1:l_1, \dots, a_n:l_n)\\
	1 \le m \le n\\
	v = S_2(l_m)
	 \end{array}}{[Array-Lookup]}
$$

\item \label{while}

Here's one way to do it. This approach literally checks if this is the
last time around the loop or not, and behaves accordingly. We also use the
old false rule to handle loops where $e_1$ is never true.

$$
\osrule{so}{S_1} E {\mbox{while }e_1\mbox{ loop } e_2\mbox{ pool}}{void}{S_{2}}
	{\begin{array}{l}
	\ossimple{so}{S_1}{E}{e_1}{Bool(false)}{S_2}\\
	 \end{array}}{[Loop-False]}
$$

$$
\osrule{so}{S_1} E {\mbox{while }e_1\mbox{ loop } e_2\mbox{ pool}}{v_2}{S_{4}}
	{\begin{array}{l}
	\ossimple{so}{S_1}{E}{e_1}{Bool(true)}{S_2}\\
	\ossimple{so}{S_2}{E}{e_2}{v_2}{S_3}\\
	\ossimple{so}{S_3}{E}{e_1}{Bool(false)}{S_4}\\
	 \end{array}}{[Loop-True-Last]}
$$

$$
\osrule{so}{S_1} E {\mbox{while }e_1\mbox{ loop } e_2\mbox{ pool}}{v_3}{S_{4}}
	{\begin{array}{l}
	\ossimple{so}{S_1}{E}{e_1}{Bool(true)}{S_2}\\
	\ossimple{so}{S_2}{E}{e_2}{v_2}{S_3}\\
	\ossimple{so}{S_3}{E}{e_1}{Bool(true)}{S_{peek}}\\
	\ossimple{so}{S_3}{E}{\mbox{while }e_1\mbox{ loop } e_2\mbox{ pool}}{v_3}{S_4}\\
	 \end{array}}{[Loop-True-Not-Last]}
$$

Note that $S_{peek}$ is just thrown away; we're just using it to make sure
that this isn't the last time around the loop.

On a real machine, we can't usually throw away a store like this; this
solution is a good example of operationaly semantics being more powerful
than we can easily implement. An alternate approach is to allocate a chunk
of memory for the value of the loop expression, and write into it each
time we traverse the loop.

\end{enumerate}

\end{document}

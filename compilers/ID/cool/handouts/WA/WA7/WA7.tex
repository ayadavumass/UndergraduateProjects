\documentclass{article}

\usepackage{../handout}
\usepackage{fullpage}
\usepackage{graphicx}
\usepackage{amssymb}

% Handout Numbers

\newcommand{\PAOneNum}{1}
\newcommand{\WAOneNum}{2}
\newcommand{\PATwoNum}{3}
\newcommand{\PAThreeNum}{4}
\newcommand{\WATwoNum}{5}
\newcommand{\WATwoSolNum}{6}
\newcommand{\WAOneSolNum}{7}
\newcommand{\WAThreeNum}{8}
\newcommand{\WAFourNum}{9}
\newcommand{\WAThreeSolNum}{10}
\newcommand{\PAFourNum}{11}
\newcommand{\WAFourSolNum}{12}
\newcommand{\WAFiveNum}{13}
\newcommand{\WAFiveSolNum}{14}
\newcommand{\WASixNum}{15}
\newcommand{\WASixSolNum}{16}
\newcommand{\PAFiveNum}{17}
\newcommand{\WASevenNum}{18}
\newcommand{\WASevenSolNum}{19}
\newcommand{\PAExtraNum}{20}
\newcommand{\WAEightNum}{21}
\newcommand{\WAEightSolNum}{22}
\newcommand{\PASixNum}{23}
\newcommand{\WANineNum}{24}
\newcommand{\WANineSolNum}{25}
\newcommand{\WATenNum}{26}
\newcommand{\WATenSolNum}{27}
\newcommand{\WAElevenNum}{28}
\newcommand{\WAElevenSolNum}{29}


\begin{document}

\newcommand{\ttmath}[1]{$\mathtt{#1}$}

\handout{\WASevenNum}{2}{Written Assignment 7 \\ Due Thursday April 1, 2004}

This assignment asks you to prepare written answers to questions on
object layout and operational semantics.
Each of the questions has a short answer. You
may discuss this assignment with other students and work on the
problems together.  However, your write-up should be your own
individual work.  Remember that written assignments are to be turned
in either at the start of lecture or in the CS164 homework box in 283
Soda by 12:30 PM on the due date.

\medskip

{\bf\em Please write your name, your account name, your TA's name, and
your section time on your homework!}  We need this information so that we
can give you credit for the assignment and so that we can return it to
you.

\bigskip

\begin{enumerate}

\item

Consider the following Cool classes:

\begin{center}
\begin{minipage}{6cm}
\begin{verbatim}
class A {
    attr1 : Int;
    attr2 : Int;
    method1() : Object { ... };
    method2() : Object { ... };
};

class B inherits A {
    attr3 : Int;
    method1() : Object { ... };
    method3() : Object { ... };
};
\end{verbatim}
\end{minipage}
\end{center}

\begin{enumerate}

\item Draw a diagram that illustrates the layout of objects of type {\tt
A} and {\tt B}, including their dispatch tables.

\item Let {\tt obj} be a variable whose static type is {\tt A}.  Assume
that {\tt obj} is stored in register {\tt \$a0}.  Write MIPS code for the
function invocation {\tt obj.method2()}.  You may use temporary registers
such as {\tt \$t0} if you wish.

\item Explain what happens in part (b) if {\tt obj} has dynamic type {\tt
B}.

\end{enumerate}

\medskip

\item 

Suppose you wish to add arrays to Cool using the following syntax:

\begin{center}
\begin{tabular}{ll}
\texttt{let a:T[\ttmath{e_1}] in \ttmath{e_2}} &
  Create an array $a$ with size $e_1$ of $T$'s, usable in $e_2$ \\
\texttt{a[\ttmath{e_1}] <- \ttmath{e_2}} &
  Assign $e_2$ to element $e_1$ in $a$ \\
\texttt{a[e]} &
  Get element $e$ of $a$ 
\end{tabular}
\end{center}

Write the operational semantics for these three syntactic constructs. You
may find it helpful to think of an array of type $T[n]$ as an object with
$n$ attributes of type $T$.

\pagebreak

\item

The operational semantics for Cool's {\tt while} expression show that
result of evaluating such an expression is always {\tt void}.  (See page
28 of the Cool manual.)

However, we could have used the following alternative semantics:

\begin{itemize}

\item If the loop body executes at least once, the result of the {\tt
while} expression is the result from the last iteration of the loop body.

\item If the loop body never executes (i.e., the condition is false the
first time it is evaluated), then the result of the {\tt while} expression
is {\tt void}.

\end{itemize}

For example, consider the following expression:

\begin{center}
{\tt while (x < 10) loop x <- x+1 pool}
\end{center}

The result of this expression would be 10 if {\tt x} $<$ 10 or {\tt void}
if {\tt x} $\geq$ 10.

Write new operational rules for the {\tt while} construct that formalize
these alternative semantics.

\end{enumerate}

\end{document}

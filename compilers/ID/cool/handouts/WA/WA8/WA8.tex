\documentclass{article}

\usepackage{../handout}
\usepackage{fullpage}
\usepackage{graphicx}
\usepackage{amssymb}

% Handout Numbers

\newcommand{\PAOneNum}{1}
\newcommand{\WAOneNum}{2}
\newcommand{\PATwoNum}{3}
\newcommand{\PAThreeNum}{4}
\newcommand{\WATwoNum}{5}
\newcommand{\WATwoSolNum}{6}
\newcommand{\WAOneSolNum}{7}
\newcommand{\WAThreeNum}{8}
\newcommand{\WAFourNum}{9}
\newcommand{\WAThreeSolNum}{10}
\newcommand{\PAFourNum}{11}
\newcommand{\WAFourSolNum}{12}
\newcommand{\WAFiveNum}{13}
\newcommand{\WAFiveSolNum}{14}
\newcommand{\WASixNum}{15}
\newcommand{\WASixSolNum}{16}
\newcommand{\PAFiveNum}{17}
\newcommand{\WASevenNum}{18}
\newcommand{\WASevenSolNum}{19}
\newcommand{\PAExtraNum}{20}
\newcommand{\WAEightNum}{21}
\newcommand{\WAEightSolNum}{22}
\newcommand{\PASixNum}{23}
\newcommand{\WANineNum}{24}
\newcommand{\WANineSolNum}{25}
\newcommand{\WATenNum}{26}
\newcommand{\WATenSolNum}{27}
\newcommand{\WAElevenNum}{28}
\newcommand{\WAElevenSolNum}{29}


\newtheorem{theorem}{Theorem}
\newcommand{\thmLabel}[1]{{\rm (}{\em #1\/}{\rm )}}

\newcommand{\ntsym}[1]{\hbox{\em #1}}
\newcommand{\tsym}[1]{\hbox{\rm #1}}

\def\ra{\rightarrow}     % grammar "rewrites-as" symbol
\def\ep{\varepsilon}     % epsilon for empty-string


\begin{document}
\handout{\WAEightNum}{1}{Written Assignment 8 \\ Due April 8, 2004}

This assignment asks you to prepare written answers to questions on
local and global optmiizations.  Each of the questions has a short
answer.  You may discuss this assignment with other students and work
on the problems together.  However, your write-up should be your own
individual work.  {\em Please write the name of the account you are
using for CS164 {\bf and your section time} on your homework.}
Remember that written assignments are to be turned in either in class
or in the CS164 homework box in 283 Soda by 11 AM on the due date.

\begin{enumerate}
\item Consider the following code:
\begin{verbatim}
  a := b + c
  z := a ** 2
  x := 0 * b
  y := b + c
  w := y * y
  u := x + 3
  v := u + w
\end{verbatim}
Assume that only \texttt{v} and \texttt{z} are live at the exit of
this block.  In order, apply the following optimizations to this basic
block.  Show the result of each transformation.
\begin{enumerate}
\item algebraic simplification
\item common sub-expression elimination
\item copy propagation
\item constant folding/propagation
\item dead code elimination
\end{enumerate}

When you've completed part (e), the resulting program will still not
be optimal.  What optimizations, in what order, can you apply to
optimize the result of (e) further?

\item Consider the following fragment of intermediate code (here
\verb'a % b' is \verb'a mod b', i.e., the remainder of dividing
\texttt{a} by \texttt{b}), and the \texttt{/} operator is integer
division:
\begin{verbatim}
    if a = 2 goto L3
L0: b := 2
L1: d := a / 2
    c := a % b
    if c = 0 goto L2
    if b >= d goto L3
    b := b + 1
    goto L1
L2: a := a + 1
    goto L0
L3:
\end{verbatim}
\begin{enumerate}
\item Draw a control-flow graph for this program.  Place each basic
block in a single node.  Include the conditionals in the basic blocks.
\item Annotate your control-flow graph with the set of variables live
before and after each statement, assuming that only \texttt{a} is live
at the entry to \texttt{L3}.
\item Describe concisely the function that this program computes?
\end{enumerate}

\end{enumerate}
\end{document}

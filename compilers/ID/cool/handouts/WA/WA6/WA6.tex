\documentclass{article}

\usepackage{../handout}
\usepackage{fullpage}
\usepackage{graphicx}
\usepackage{amssymb}

% Handout Numbers

\newcommand{\PAOneNum}{1}
\newcommand{\WAOneNum}{2}
\newcommand{\PATwoNum}{3}
\newcommand{\PAThreeNum}{4}
\newcommand{\WATwoNum}{5}
\newcommand{\WATwoSolNum}{6}
\newcommand{\WAOneSolNum}{7}
\newcommand{\WAThreeNum}{8}
\newcommand{\WAFourNum}{9}
\newcommand{\WAThreeSolNum}{10}
\newcommand{\PAFourNum}{11}
\newcommand{\WAFourSolNum}{12}
\newcommand{\WAFiveNum}{13}
\newcommand{\WAFiveSolNum}{14}
\newcommand{\WASixNum}{15}
\newcommand{\WASixSolNum}{16}
\newcommand{\PAFiveNum}{17}
\newcommand{\WASevenNum}{18}
\newcommand{\WASevenSolNum}{19}
\newcommand{\PAExtraNum}{20}
\newcommand{\WAEightNum}{21}
\newcommand{\WAEightSolNum}{22}
\newcommand{\PASixNum}{23}
\newcommand{\WANineNum}{24}
\newcommand{\WANineSolNum}{25}
\newcommand{\WATenNum}{26}
\newcommand{\WATenSolNum}{27}
\newcommand{\WAElevenNum}{28}
\newcommand{\WAElevenSolNum}{29}


\newtheorem{theorem}{Theorem}
\newcommand{\thmLabel}[1]{{\rm (}{\em #1\/}{\rm )}}

\newcommand{\ntsym}[1]{\hbox{\em #1}}
\newcommand{\tsym}[1]{\hbox{\rm #1}}

\def\ra{\rightarrow}     % grammar "rewrites-as" symbol
\def\ep{\varepsilon}     % epsilon for empty-string


\begin{document}
\handout{\WASixNum}{3}{Written Assignment 6 \\ Due March 18, 2004}

This assignment asks you to prepare written answers to questions on
run-time environment and code generation. Each of the questions has a short answer. You
may discuss this assignment with other students and work on the
problems together.  However, your write-up should be your own
individual work.  Remember that written assignments are to be turned
in either at the start of lecture or in the CS164 homework box in 283
Soda by 12:30 PM on the due date.

\medskip

{\bf\em Please write your name, your account name, your TA's name, and
your section time on your homework!}  We need this information so that we
can give you credit for the assignment and so that we can return it to
you.

\bigskip

\begin{enumerate}

\item Suppose \texttt{f} is a function with a call to \texttt{g}
somewhere in the body of \texttt{f}:
\begin{verbatim}
f(...) {
 ... g(...) ...
}
\end{verbatim}
We say that this particular call to \texttt{g} is a {\it tail call}
if the call is the last thing \texttt{f} does before returning.  For
example, consider the following functions for computing positive
powers of 2:
\begin{verbatim}
f(x:Int, acc:Int) : Int { if x > 0 then f(x-1, acc*2) else acc fi };
g(x:Int) : Int { if x > 0 then 2*g(x-1) else 1 fi };
\end{verbatim}
Here $\verb'f(x, 1)' = \verb'g(x)' = 2^x$ for $x\geq 0$.  The
recursive call to \texttt{f} is a tail call, while the recursive call
to \texttt{g} is not.  A function in which all recursive calls are
tail calls is called {\it tail recursive}.

\begin{enumerate}
\item Here is a non-tail recursive function for computing factorial:
\begin{verbatim}
fact(n:Int) : Int { if n > 0 then n*fact(n-1) else 1 fi };
\end{verbatim}
Write a tail recursive function \verb'fact2' that computes the same
result.  (Hint: Your function will most likely need two arguments, or
it may need to invoke a function of two arguments.)

\item Recall from lecture that function calls are usually implemented
using a stack of activation records.  Trace through the execution of
\verb'fact' and \verb'fact2' both computing $4!$, writing out the
stack of activation records at each step (i.e., draw the tree of
activation records).  (If you were unable to write a tail-recursive
version of \verb'fact', show functions \texttt{f} and \texttt{g} from
above computing $2^4$.)  Indicate the amount of computation done
before, during, and after each record is created or destroyed.  Is
there any place where you can see potential for making the execution
of the tail-recursive \verb'fact2' more time- or space-efficient than
\verb'fact' (without changing \verb'fact2''s source code)?  What could
you do?

%Suppose we have some way of
%recognizing that \verb'fact2' is tail-recursive.  How can a compiler
%take advantage of this fact?  (Hint: Compare the stack space required
%for \verb'fact(99)' with the stack space required for
%\verb'fact2(99)'.  Can \verb'fact2' use fewer activation records?)

\item Now consider the following pair of functions:
\begin{verbatim}
f(x:Int, acc:Int) : Int { if x > 0 then g(x-1, acc*2) else acc fi };
g(x:Int, acc:Int) : Int { if x > 0 then f(x-1, acc*5) else acc fi };
\end{verbatim}
In this case, the calls to \texttt{g} and \texttt{f} are all tail
calls but they are not immediately recursive.  Can you extend you
answer to part (b) so that a compiler can use only one or two
activation records for a call to \texttt{f} or \texttt{g}?  (Hint:
Consider the case when the initial invocation of these functions is
via a call to \texttt{f} and the case when the initial invocation is
via a call to \texttt{g}.)
\end{enumerate}

\item Consider the following MIPS assembly code program.  Using the
code generation rules from lecture, what source program produces this
code?
\begin{verbatim}
f_entry:
        move  $fp $sp
        sw    $ra 0($sp)
        addiu $sp $sp -4
        lw    $a0 4($fp)
        sw    $a0 0($sp)
        addiu $sp $sp -4
        li    $a0 0
        lw    $t1 4($sp)
        addiu $sp $sp 4
        beq   $a0 $t1 true_branch
false_branch:
        lw    $a0 4($fp)
        sw    $a0 0($sp)
        addiu $sp $sp -4
        sw    $fp 0($sp)
        addiu $sp $sp -4
        lw    $a0 4($fp)
        sw    $a0 0($sp)
        addiu $sp $sp -4
        li    $a0 1
        lw    $t1 4($sp)
        sub   $a0 $t1 $a0
        addiu $sp $sp 4
        sw    $a0 0($sp)
        addiu $sp $sp -4
        jal   f_entry
        lw    $t1 4($sp)
        add   $a0 $a0 $t1
        addiu $sp $sp 4
        b     end_if
true_branch:
        li    $a0 0
end_if:
        lw    $ra 4($sp)
        addiu $sp $sp 12
        lw    $fp 0($sp)
        jr    $ra
\end{verbatim}

\item Give a recursive definition of the cgen function (as given in
  slide 52 in the lecture notes) for the following new construct. 
\begin{eqnarray*}
\tt{for}\;\; i = e_1 \;\;to\;\; e_2 \;\;\tt{by}\;\; e_3 \;\;\tt{do }\;\; e_4
\end{eqnarray*}

Assume that the subexpressions $e_1, e_2, e_3$ and $e_4$ are
integer-valued. A ``for loop'' expression is evaluated according to
the following rules. The first three subexpressions are evaluated once 
at the start of the loop in the order $e_1$, $e_2$, and then $e_3$. 
The subexpression $e_4$ is evaluated once per iteration of the loop.
The index variable $i$ is initialized to the value of $e_1$. 
The loop bound is the value of $e_2$ and $i$ is incremented by the 
value of $e_3$ after each iteration. The loop terminates before
executing an iteration where the value of $i$ is greater than the 
loop bound. The value returned by the ``for loop'' expression is the value of the
expression $e_4$ in the last iteration. If the loop does not execute
at all, then the value returned is the integer $0$. 

Following is a more formal semantics of the for expression in terms of the Cool
expressions. 
\begin{tabbing}
  \hspace*{3mm} \= let t: Int $\leftarrow$ $e_1$ in \\
  \> let bound:Int  $\leftarrow$ $e_2$ in \\
  \> let incr:Int  $\leftarrow$ $e_3$ in \\
  \> let result:Int  $\leftarrow$ $0$ in \\
  \> let \= i:Int $\leftarrow$ $t$ in   \\
   \>  \> while \= ($i$ $\leq$ bound) loop \{ \\
     \>\>\>   result $\leftarrow$ $e_4$; \\
\> \> \> $i$ $\leftarrow$ $i$ + incr; \\
     \> \> \} pool; \\
     \> \> result  \\
\end{tabbing}

Note that the expressions $e_1$, $e_2$ and $e_3$ are evaluated ONLY once before the start of the loop.
Also note that any occurences of variable $i$ in $e_1$, $e_2$ and
$e_3$ refer to the value of $i$ 
just before the for loop.
Any occurrence of variable $i$ in expression $e_4$ refers to the loop index variable $i$.


\end{enumerate}


\end{document}

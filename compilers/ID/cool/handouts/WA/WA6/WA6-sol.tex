\documentclass{article}

\usepackage{../handout}
\usepackage{proof}
% Handout Numbers

\newcommand{\PAOneNum}{1}
\newcommand{\WAOneNum}{2}
\newcommand{\PATwoNum}{3}
\newcommand{\PAThreeNum}{4}
\newcommand{\WATwoNum}{5}
\newcommand{\WATwoSolNum}{6}
\newcommand{\WAOneSolNum}{7}
\newcommand{\WAThreeNum}{8}
\newcommand{\WAFourNum}{9}
\newcommand{\WAThreeSolNum}{10}
\newcommand{\PAFourNum}{11}
\newcommand{\WAFourSolNum}{12}
\newcommand{\WAFiveNum}{13}
\newcommand{\WAFiveSolNum}{14}
\newcommand{\WASixNum}{15}
\newcommand{\WASixSolNum}{16}
\newcommand{\PAFiveNum}{17}
\newcommand{\WASevenNum}{18}
\newcommand{\WASevenSolNum}{19}
\newcommand{\PAExtraNum}{20}
\newcommand{\WAEightNum}{21}
\newcommand{\WAEightSolNum}{22}
\newcommand{\PASixNum}{23}
\newcommand{\WANineNum}{24}
\newcommand{\WANineSolNum}{25}
\newcommand{\WATenNum}{26}
\newcommand{\WATenSolNum}{27}
\newcommand{\WAElevenNum}{28}
\newcommand{\WAElevenSolNum}{29}

\usepackage{fullpage}
\usepackage{graphicx}
\usepackage{amssymb}


\newtheorem{theorem}{Theorem}
\newcommand{\thmLabel}[1]{{\rm (}{\em #1\/}{\rm )}}

\newcommand{\ntsym}[1]{\hbox{\em #1}}
\newcommand{\tsym}[1]{\hbox{\rm #1}}
\newcommand{\infertext}[2]{\infer{{\textrm{#1}}}{#2}}

\def\ra{\rightarrow}     % grammar "rewrites-as" symbol
\def\ep{\varepsilon}     % epsilon for empty-string


\begin{document}
\handout{\WASixSolNum}{2}{Solutions to Written Assignment 6}

\begin{enumerate}

\item 
\begin{enumerate}
\item Here is a tail-recursive function that computes factorial:
\begin{verbatim}
fact2(n:Int, acc:Int) : Int { if n > 0 then fact2(n-1, n*acc) else acc fi };
\end{verbatim}
Here \verb'fact2(n, 1)' = $n!$.  Alternately, if you wanted
\verb'fact2' to have only one parameter, you could have named the
above function \verb'fact3' and defined
\verb'fact2(n:Int) : Int { fact3(n,1) };'

\item If we trace out the execution of \verb'fact' and \verb'fact2'
computing $4!$, we see that they both require at most five activation
records on the stack.  Below we show the stack during the evaluation
of the last recursive call:

\begin{center}
\begin{tabular}{|c|c|c|} 
\cline{1-1} \cline{3-3}
AR for \verb'fact(4)' & \hspace*{1in} & AR for \verb'fact2(4, 1)' \\
\cline{1-1} \cline{3-3}
AR for \verb'fact(3)' && AR for \verb'fact2(3, 4)' \\
\cline{1-1} \cline{3-3}
AR for \verb'fact(2)' && AR for \verb'fact2(2, 12)' \\
\cline{1-1} \cline{3-3}
AR for \verb'fact(1)' && AR for \verb'fact2(1, 24)' \\
\cline{1-1} \cline{3-3}
AR for \verb'fact(0)' && AR for \verb'fact2(0, 24)' \\
\cline{1-1} \cline{3-3}
\end{tabular}
\end{center}

For \verb'fact', before a new activation record is pushed onto the
stack we do one subtraction (decrease \texttt{n} by one).  Just before
we remove an activation record from the stack (i.e., just before we
return) we do one multiplication (multiply \texttt{n*fact(n-1)}).  For
\verb'fact2', we do the subtraction and multiplication before each new
record is pushed onto the stack.  When we remove an activation record
from the stack all we do is take the result of the recursive call to
\verb'fact2' and return it.  Thus the only part of the stack frame for
\verb'fact2' we are using after a recursive call is the return
address.

We can implement \verb'fact2' more efficiently by reusing the same
activation record for a recursive call, rather than pushing a new
record on the stack.  To recursively call \verb'fact2', we compute new
values for \texttt{n} and \texttt{a} in temporary space, then to do
the function call we replace \texttt{n} and \texttt{a} on the stack
with their new values and restart \verb'fact2'.

On the MIPS architecture, there are at least two ways to re-enter
\verb'fact2'.  One choice is to retrieve the return address from the
stack, store it in \texttt{\$ra}, pop everything except the parameters
off the stack, and then unconditionally jump to \verb'fact2''s entry
point (we don't want to clobber \texttt{\$ra}).

Another choice is to notice that the correct frame pointer and return
address are already on the stack, and so we pop everything up to the
return address off the stack, and then unconditionally jump just past
\verb'fact2''s entry point to skip the initial set-up code.

In either case, no new space is required.  With our new
implementation, \verb'fact2(n)' runs in constant space for any
\texttt{n}, whereas \verb'fact(n)' requires $O(\texttt{n})$ space.

% typo fixed
%\item There was a minor typo on this problem.  Both \texttt{f} and
%\texttt{g} should return \texttt{acc}, not \texttt{y}, in the base
%case.

For mutually tail-recursive \texttt{f} and \texttt{g} we need to be a
little more careful than in part (b).  In this case, \texttt{f} and
\texttt{g} have the same number of arguments, so we could compile
\texttt{f} and \texttt{g} into a single block of code with two entry
points $f$ and $g$.  Then a function \texttt{h} that calls either
\texttt{f} or \texttt{g} pushes two arguments onto the stack, stores
the return address in \texttt{\$ra} (on MIPS) and then jumps to either
$f$ or $g$.  \texttt{f} or \texttt{g} modifies the arguments
appropriately and then jumps to the other function (using one of the
two strategies described above), or returns \texttt{acc} in the base
case.

In general, however, \texttt{f} and \texttt{g} might have different
signatures (e.g., take different numbers of parameters) but still be
mutually tail-recursive, so we need a more general strategy.

To implement a general tail call to some function \texttt{h} we need
to replace the current activation record with a new activation record
that is correctly layed out for \texttt{h}.  As before we compute the
arguments to \texttt{h} in temporary space.  Then we retreive the
return address from the stack (on the MIPS we store it in
\texttt{\$ra}).  Finally we overwrite the current activation record
with the new parameters, shifting the stack pointer just past the last
new parameter, and jump unconditionally to \texttt{h}.
\end{enumerate}

\item The program that generates this code is
\[
\verb'def f(n) = if n=0 then 0 else n+f(n-1)'
\]


\item 

\begin{verbatim}
cgen(for i = e1 to e2 by e3 do e4, nt) = 
           cgen(e1, nt) 
           sw $a0 -nt*4($fp)  
           cgen(e2, nt+1) 
           sw $a0 -(nt+1)*4($fp) 
           cgen(e3, nt+2) 
           sw $a0 -(nt+2)*4($fp) 
           li $a0 0 
           sw $a0 -(nt+3)*4($fp)
	           ld $a0 -nt*4($fp)
     loop: ld $t1 -(nt+1)*4($fp) 
           bg $a0 $t1 finish
           cgen(e4, nt+4)
           sw $a0 -(nt+3)*4($fp)
	           ld $a0 -nt($fp) 
           ld $t1 -(nt+2)*4($fp)
           add $a0 $a0 $t1 
           sw $a0 -nt*4($fp)
           j loop 
   finish: ld $a0 -(nt+3)*4($fp)  
\end{verbatim} 

\medskip

Note that we assume that {\tt nt} starts at 1 (at the beginning of a
method) and increases by 1 for each temporary allocated (at the recursive
{\tt cgen} calls).  In lecture, we started at 4 and incremented by 4 for
each temporary allocated, but we do not multiply by 4 when computing the
address of each temporary.

\end{enumerate}

\end{document}

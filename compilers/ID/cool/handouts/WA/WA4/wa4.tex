\documentclass[11pt]{article}

\usepackage{../handout}

% Side margins:
% Actual margin is 1 in + this number
\oddsidemargin -0.25in
\evensidemargin -0.25in

% Text width:
\textwidth 6.9in

% Top margin:
% Actual margin is 1.5 in + this number
\topmargin -.3in

% Text height:
\textheight 8.7in


\begin{document}
\handout{11}{2}{Written Assignment 4 \\
Due Tuesday, March 13, 2007}


This assignment asks you to prepare written answers to questions on
code generation, operational semantics, optimization, register
allocation, and garbage collection.  Each of the questions has a short
answer.  You may discuss this assignment with other students and work
on the problems together.  However, your write-up should be your own
individual work.  Written assignments can be turned in at the start of
lecture.  Alternatively, assignments can be turned in at Professor
Aiken's office in Gates 411, or submitted electronically in PDF format
by following the electronic submission instructions at
\texttt{http://www.stanford.edu/class/cs143/policies/submit.html}, by
5:00 PM on the due date.

%% \bigskip
%% \bigskip

\begin{enumerate}

\item
\label{cond}
Suppose that we want to add the following conditional expression to
Cool.
\vspace{-0.5\baselineskip}
\begin{verbatim}
cond <p1> => <e1>; <p2> => <e2>; ... ; <pn> => <en>; dnoc
\end{verbatim}

\vspace{-0.5\baselineskip}
There must be at least one predicate and expression pair (that is,
$n \geq 1$).  The evaluation of a \texttt{cond} expression begins with
the evaluation of the predicate \texttt{<p1>}, which must have static
type \texttt{Bool}.  If \texttt{<p1>} evaluates to \texttt{true}, then
\texttt{<e1>} is evaluated, and the evaluation of the \texttt{cond}
expression is complete.  If \texttt{<p1>} evaluates to \texttt{false},
then \texttt{<p2>} is evaluated, and this process is repeated until
one of the predicates evaluates to \texttt{true}.  The value of the
\texttt{cond} expression is the value of the expression \texttt{<ei>}
corresponding to the first predicate \texttt{<pi>} that evaluates to
\texttt{true}.  If all the predicates evaluate to \texttt{false}, then
the value of the \texttt{cond} expression is \texttt{void}.

Write operational semantics rules for this conditional expression in
Cool.

\item
Write a code generation function
\texttt{cgen(cond <p1> => <e1>; ... ; <pn> => <en>; dnoc)} for the
conditional expression described in Question \ref{cond}.  For
concreteness, assume that $n = 2$.  Use the stack machine architecture
and conventions from the lectures.

\item
Consider the following basic block, in which all variables are
integers, and \texttt{**} denotes exponentiation.
\begin{verbatim}
  a := b + c
  z := a ** 2
  x := 0 * b
  y := b + c
  w := y * y
  u := x + 3
  v := u + w
\end{verbatim}
Assume that the only variables that are live at the exit of this block
are \texttt{v} and \texttt{z}.  In order, apply the following
optimizations to this basic block.  Show the result of each
transformation.
\begin{enumerate}
\item algebraic simplification
\item common sub-expression elimination
\item copy propagation
\item constant folding
\item dead code elimination
\end{enumerate}

When you've completed part (e), the resulting program will still not
be optimal.  What optimizations, in what order, can you apply to
optimize the result of (e) further?

\item
Consider the following program.
\vspace{-0.5\baselineskip}
\begin{verbatim}
L0: e := 0
    b := 1
    d := 2
L1: a := b + 2
    c := d + 5
    e := e + c
    f := a * a
    if f < c goto L3
L2: e := e + f
    goto L4
L3: e := e + 2
L4: d := d + 4
    b := b - 4
    if b != d goto L1
L5:
\end{verbatim}

\vspace{-0.5\baselineskip}
This program uses six temporaries, \texttt{a}-\texttt{f}.  Assume that
the only variable that is live on exit from this program is
\texttt{e}.

Draw the register interference graph.  (Drawing a control-flow graph
and computing the sets of live variables at every program point may be
helpful.)

\item
Suppose that the following Cool program is executed.
\vspace{-0.5\baselineskip}
\begin{verbatim}
class C {
  x : C; y : C;
  setx(newx : C) : C { x <- newx };
  sety(newy : C) : C { y <- newy };
  setxy(newx : C, newy : C) : SELF_TYPE { { x <- newx; y <- newy; self; } }; };
class Main {
  x : C;
  main() : Object {
    let a : C <- new C, b : C <- new C, c : C <- new C, d : C <- new C,
        e : C <- new C, f : C <- new C, g : C <- new C, h : C <- new C in {
      f.sety(g); a.setxy(e, c); b.setx(f); g.setxy(f, d); c.sety(h); h.setxy(e, a);
      x <- c;
  } }; };
\end{verbatim}

\vspace{-0.75\baselineskip}
\begin{enumerate}
\item
Draw the heap at the end of the execution of the program, identifying
objects by the names to which they are bound in the \texttt{let}
expression.  Assume that the root is the \texttt{Main} object created
at the start of the program, and that this object is not in the heap.
If the value of an attribute is \texttt{void}, don't show that
attribute as a pointer on the diagram.

\item
For each of the garbage collection algorithms discussed in class (Mark
and Sweep, Stop and Copy, and Reference Counting), show the heap after
garbage collection.  When the pointers of an object are processed,
assume that the processing order is the order of the attributes in the
source program.  Assume that the heap has space for at least $16$
objects.
\end{enumerate}

\end{enumerate}

\end{document}

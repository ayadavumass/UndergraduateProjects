\documentclass[11pt]{article}

\usepackage{../handout}
%\usepackage{upquote}
%\usepackage{graphicx, color}

% Side margins:
% Actual margin is 1 in + this number
\oddsidemargin -0.25in
\evensidemargin -0.25in

% Text width:
\textwidth 6.9in

% Top margin:
% Actual margin is 1.5 in + this number
\topmargin -.3in

% Text height:
\textheight 8.7in


\begin{document}
\handout{8}{4}{Written Assignment 3 \\
Due Tuesday, February 27, 2007}


This assignment asks you to prepare written answers to questions on
type checking, run-time environments, and code generation.  Each of
the questions has a short answer.  You may discuss this assignment
with other students and work on the problems together.  However, your
write-up should be your own individual work.  Written assignments can
be turned in at the start of lecture.  Alternatively, assignments can
be turned in at Professor Aiken's office in Gates 411, or submitted
electronically in PDF format by following the electronic submission
instructions at
\texttt{http://www.stanford.edu/class/cs143/policies/submit.html}, by
5:00 PM on the due date.

\bigskip
\bigskip

\begin{enumerate}

\item Consider the following class definitions.
\begin{verbatim}
class A {
  i : Int;
  o : Object;
  a : A <- new B;
  b : B <- new B;
  x : SELF_TYPE;
  f() : SELF_TYPE { x };
};
class B inherits A {
  g(b : Bool) : Object { (* EXPRESSION *) };
};
\end{verbatim}

Assume that the type checker implements the rules described in the
lectures and in the Cool Reference Manual.  For each of the following
expressions, occurring in place of \texttt{(* EXPRESSION *)} in the
body of the method \texttt{g}, show the static type inferred by the
type checker for the expression.  If the expression causes a type
error, give a brief explanation of why the appropriate type checking
rule for the expression cannot be applied.
\begin{verbatim}
1)  i + i
2)  x
3)  self = x
4)  self = i
5)  let x : B <- x in x
6)  case o of
        o : Int => b;
        o : Bool => o;
        o : Object => true;
    esac
7)  a.f().g(b)
8)  f()
\end{verbatim}

\item Someone has proposed that Cool be extended to allow comparison,
addition, and multiplication operations on \texttt{Bool} objects as
well as on \texttt{Int} objects.  The comparison, addition, and
multiplication operations are now defined for any combination of
\texttt{Int} and \texttt{Bool} operands.  An addition or
multiplication operation involving an operand of type \texttt{Bool}
produces a result of type \texttt{Int} (the \texttt{Bool} object is
converted to $1$ if it has the value \texttt{true}, and to $0$ if it
has the value \texttt{false}).

Write the additional type checking rules (as in the lecture and the
Cool Reference Manual) for these operations on \texttt{Bool} objects.

\item
\begin{enumerate}
\item Your friend Damon feels constrained by the fact that every
\texttt{while} expression in Cool evaluates to \texttt{void}.  He
would like to be able to write functions such as this:
\begin{verbatim}
f() : Bool {
  let x : Bool <- true in
    while x loop x <- false pool
};
\end{verbatim}
Damon wants to change the semantics of the \texttt{while} expression
so that the value of the expression is the value of the body on the
last execution of the loop.  He must define the value of a
\texttt{while} expression in the case that the body of the loop is
never evaluated, however.

Damon's first proposal is to define the value of the \texttt{while}
expression to be \texttt{void} when the predicate of the loop is
\texttt{false} initially and the body is never evaluated.  He says
that the type checker can now infer the static type of the
\texttt{while} expression to be the static type of the body, because
\texttt{void} is a member of every type.  Give an example Cool program
that shows how this change would cause a runtime type error (beyond
the existing Cool runtime errors), and explain how the error occurs.

\item After seeing your example, Damon comes up with a new suggestion.
He proposes to eliminate the requirement that the predicate expression
in a loop must be of static type \texttt{Bool}.  Now, the predicate
can have any static type, and a new method \texttt{is\_true() : Bool}
is added to the \texttt{Object} class.

The predicate is evaluated before each iteration of the loop.  If the
value of the predicate is \texttt{void}, the loop terminates.
Otherwise, the method \texttt{is\_true()} of the value of the
predicate is invoked, and the loop terminates if the value returned by
the method is \texttt{false}.  If the value returned by the method is
\texttt{true}, then the body of the loop is evaluated, and the process
repeats.  The value of the \texttt{while} expression is determined as
follows:

\begin{itemize}
\item If the body of the loop is never evaluated, then the value of
the \texttt{while} expression is the value of the predicate (from the
first evaluation of the predicate).

\item Otherwise, the value of the \texttt{while} expression is the
value of the body on the last execution of the loop.
\end{itemize}

Can these modified \texttt{while} semantics be type checked statically
to accept Damon's sample function above, while ensuring type safety
(i.e., that no runtime type error will occur)?  If so, write the most
flexible type rule (the rule that accepts the most correct programs)
for the modified \texttt{while} expression.  If not, explain why not,
and give an example Cool program that illustrates how this expression
can introduce new runtime errors (beyond the existing Cool runtime
errors).

\item Damon wants to extend Cool by allowing method assignments.  He
would like to add a new assignment expression of the following form.
\begin{verbatim}
<exprB>.g <- <exprA>.f
\end{verbatim}
\noindent
Suppose that \texttt{<exprA>} evaluates to an object \texttt{a} of
class \texttt{A}, and \texttt{<exprB>} evaluates to an object
\texttt{b} of class \texttt{B}.  Furthermore, \texttt{A} has a method
named \texttt{f} and \texttt{B} has a method named \texttt{g}, and
the two methods \texttt{f} and \texttt{g} have the same signature (the
signature consists of the number of arguments, the types of the formal
parameters, and the return type).  The effect of the assignment would
be to set the body of the method \texttt{g} of the object \texttt{b}
to the body of the method \texttt{f} of the object \texttt{a}, so that
subsequent invocations of the method \texttt{g} belonging to
\texttt{b} would execute the body of the method \texttt{f} belonging
to \texttt{a}.  The value of the assignment expression would be
\texttt{void}.

Damon says that, if \texttt{B} is a subclass of \texttt{A} (a
descendant of \texttt{A} in the inheritance graph), then the
inheritance rules of Cool guarantee that this operation is type safe.

Can Damon's method assignment expression be type checked statically to
guarantee type safety?  If so, write the most flexible type rule for
the method assignment expression.  If not, explain why not, and give
an example Cool program that illustrates how this expression can
introduce new runtime errors (beyond the existing Cool runtime
errors).
\end{enumerate}

\item Suppose \texttt{f} is a function with a call to \texttt{g}
somewhere in the body of \texttt{f}.
\begin{verbatim}
f(...) {
 ... g(...) ...
}
\end{verbatim}
We say that this particular call to \texttt{g} is a {\it tail call}
if the call is the last thing \texttt{f} does before returning.  For
example, consider the following two functions for computing positive
powers of $2$.
\begin{verbatim}
f(x : Int, acc : Int) : Int { if (0 < x) then f(x - 1, acc * 2) else acc fi };
g(x : Int) : Int { if (0 < x) then (2 * g(x - 1)) else 1 fi };
\end{verbatim}
Here $\verb'f(x, 1)' = \verb'g(x)' = 2^x$ for $x\geq 0$.  The
recursive call to \texttt{f} is a tail call, while the recursive call
to \texttt{g} is not.  A function in which all recursive calls are
tail calls is called {\it tail recursive}.

\begin{enumerate}
\item Here is a non-tail recursive function for computing factorials.
\begin{verbatim}
fact(n : Int) : Int { if (0 < n) then (n * fact(n - 1)) else 1 fi };
\end{verbatim}
Write a tail recursive function \texttt{fact2} that computes the same
result.  (Hint: Your function will most likely need two arguments, or
it may need to invoke a function of two arguments.)

\item Recall from lecture that function calls are usually implemented
using a stack of activation records.  Trace through the execution of
\texttt{fact} and \texttt{fact2} as they compute $4!$, showing the
tree of activation records (each node of the tree shows the invocation
of a function, and the arguments).  How can a compiler make the
execution of the tail recursive function \texttt{fact2} more efficient
than that of \texttt{fact}?  (Hint: Compare the stack space required
for \texttt{fact(99)} with the stack space required for
\texttt{fact2(99)}.  Can \texttt{fact2} use fewer activation records?)
\end{enumerate}

\item In some languages, a class can have multiple methods with the
same name, as long as these methods differ in the number and/or types
of formal parameters.  This is referred to as method
{\em overloading}.

Suppose we would like to add method overloading to Cool.  Now, when
generating code for a dispatch expression
$e_{0}.f(e_{1}, \dots, e_{n})$, the compiler may need to choose the
method to dispatch to (i.e., which slot in the dispatch table to jump
to) amongst several valid possibilities.  Let $T_{i}$ be the static
type of $e_{i}$ for $i = 0, 1, \dots, n$.  Suppose that the compiler
chooses a method $f$ for the dispatch such that:
\begin{itemize}
\item $T_{0}$ has a method $f$ with $n$ formal parameters of types
$P_{1}, \dots, P_{n}$; and

\item $T_{i} \leq P_{i}$ for $i = 1, \dots, n$; and

\item If $T_{0}$ has more than one method named $f$, then, for any
other method named $f$ with $n$ formal parameters
$Q_{1}, \dots, Q_{n}$ satisfying $T_{i} \leq Q_{i}$ for
$i = 1, \dots, n$, it must be the case that $P_{i} \leq Q_{i}$ for
$i = 1, \dots, n$.  In other words, $P_{1}, \dots, P_{n}$ are the most
specific parameter types for a method named $f$ that could be invoked.
\end{itemize}

If a {\em unique} method exists under these rules, then the dispatch
is accepted by the type checker.  If more than one method satisfies
these conditions, then the type checker signals a type error at
compile time.

Method overriding occurs as described in the original Cool Reference
Manual.  Specifically, a method defined in a child class overrides any
method with the identical signature in the parent class.

Consider the following Cool program:
\begin{verbatim}
class A inherits IO {
  f(a : Object, b : Object) : Object { out_string("1") };
  f(a : Object, b : Int) : Object { out_string("2") };
};
class B inherits A {
  f(a : Object, b : Object) : Object { out_string("3") };
  f(a : Int, b : Object) : Object { out_string("4") };
};
class Main {
  main() : Object {
    let a : A <- new B,
        b : B <- new B,
        x : Object <- new Object,
        y : Object <- 1,
        z : Int <- 2 in
      (* DISPATCH *)
  };
};
\end{verbatim}

For each of the following dispatch expressions, give the output of the
program when \texttt{(* DISPATCH *)} is replaced by the dispatch
expression, or specify that a type error would occur.
\begin{center}
\begin{tabular*}{0.5\textwidth}{@{\extracolsep{\fill}}ll}
\texttt{a.f(x, x)} & \texttt{b.f(x, x)} \\
\texttt{a.f(x, y)} & \texttt{b.f(x, y)} \\
\texttt{a.f(x, z)} & \texttt{b.f(x, z)} \\
\texttt{a.f(y, x)} & \texttt{b.f(y, x)} \\
\texttt{a.f(y, y)} & \texttt{b.f(y, y)} \\
\texttt{a.f(y, z)} & \texttt{b.f(y, z)} \\
\texttt{a.f(z, x)} & \texttt{b.f(z, x)} \\
\texttt{a.f(z, y)} & \texttt{b.f(z, y)} \\
\texttt{a.f(z, z)} & \texttt{b.f(z, z)}
\end{tabular*}
\end{center}

\end{enumerate}

\end{document}

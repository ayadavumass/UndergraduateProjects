\documentclass{article}

\usepackage{../handout}
\usepackage{fullpage}
\usepackage{graphicx}
\usepackage{amssymb}

% Handout Numbers

\newcommand{\PAOneNum}{1}
\newcommand{\WAOneNum}{2}
\newcommand{\PATwoNum}{3}
\newcommand{\PAThreeNum}{4}
\newcommand{\WATwoNum}{5}
\newcommand{\WATwoSolNum}{6}
\newcommand{\WAOneSolNum}{7}
\newcommand{\WAThreeNum}{8}
\newcommand{\WAFourNum}{9}
\newcommand{\WAThreeSolNum}{10}
\newcommand{\PAFourNum}{11}
\newcommand{\WAFourSolNum}{12}
\newcommand{\WAFiveNum}{13}
\newcommand{\WAFiveSolNum}{14}
\newcommand{\WASixNum}{15}
\newcommand{\WASixSolNum}{16}
\newcommand{\PAFiveNum}{17}
\newcommand{\WASevenNum}{18}
\newcommand{\WASevenSolNum}{19}
\newcommand{\PAExtraNum}{20}
\newcommand{\WAEightNum}{21}
\newcommand{\WAEightSolNum}{22}
\newcommand{\PASixNum}{23}
\newcommand{\WANineNum}{24}
\newcommand{\WANineSolNum}{25}
\newcommand{\WATenNum}{26}
\newcommand{\WATenSolNum}{27}
\newcommand{\WAElevenNum}{28}
\newcommand{\WAElevenSolNum}{29}


\newtheorem{theorem}{Theorem}
\newcommand{\thmLabel}[1]{{\rm (}{\em #1\/}{\rm )}}

\newcommand{\ntsym}[1]{\hbox{\em #1}}
\newcommand{\tsym}[1]{\hbox{\rm #1}}

\def\ra{\rightarrow}     % grammar "rewrites-as" symbol
\def\ep{\varepsilon}     % epsilon for empty-string


\begin{document}
\handout{\WAFiveNum}{2}{Written Assignment 5 \\ Due March 11, 2004}

This assignment asks you to prepare written answers to questions on
type checking. Each of the questions has a short answer. You
may discuss this assignment with other students and work on the
problems together.  However, your write-up should be your own
individual work.  Remember that written assignments are to be turned
in either at the start of lecture or in the CS164 homework box in 283
Soda by 12:30 PM on the due date.

\medskip

{\bf\em Please write your name, your account name, your TA's name, and
your section time on your homework!}  We need this information so that we
can give you credit for the assignment and so that we can return it to
you.

\bigskip

\begin{enumerate}

% #1
\item Show the full type derivation (as done in slide 49 in the
  lecture notes) for the following judgement:

\begin{center}
$O$[Bool/x] $\vdash$ \tt{x $<$- (let x:Object $<$- x in x = x): Bool}
\end{center}


% 2
\item Suppose we extend the grammar for Cool with a ``{\bf void}'' keyword
\begin{eqnarray*}
  expr & ::= & {\bf void} \\
       & |   & ...
\end{eqnarray*}

  that is analogous to {\tt null} in Java. (Currently objects are
  initialized to void if they have no other initializer specified, but
  there is no general-purpose {\bf void} keyword.)  We want to be able
  to use {\bf void} whereever an object can be used, as in
\begin{verbatim}
  let foo:Int <- if some_test 
                 then 5 
                 else void 
                 fi 
  in ...
\end{verbatim}

  Give a sound typing rule that we can add to the Cool specification
  to accomodate this new keyword.

\item Suppose we extend Cool with exceptions by adding two new constructs 
to the Cool language.  
\begin{eqnarray*}
  expr & ::= & \textit{\bf try } expr \;\textit{\bf catch } ID => expr \\
       & |   & \textit{\bf throw } expr \\
       & | & ...
\end{eqnarray*}
Here {\bf try}, {\bf catch} and {\bf throw} are three new terminals. 
  ``${\bf throw }\;\; expr$'' returns $expr$ to the
  closest dynamically enclosing catch block. 
Note that since {\bf throw} expression returns control to a different location, we do not really
  care about the context in which throw is used. For example, 
({\bf throw} $false) + 2$ is a valid Cool expression (However, note that
  ({\bf throw} $false) + (2+true)$ is not a valid Cool expression).  Following is an
  example that uses the try-catch and throw constructs. 
\newpage
\begin{verbatim}
try 
   if some_test1 then throw 34
   else if some_test2 then throw ``undefined error''
   else do_something fi fi
catch x =>
  case x of 
      x:Int => do_something1
      x:String => do_something2
  esac
\end{verbatim}

The above program fragment executes
  ``do\_something1'' (with $x$ bound to the value 34) if ``some\_test1''
  evaluates to $true$. It executes ``do\_something2'' (with $x$ bound to the
  value ``undefined error'') if ``some\_test1'' evaluates to $false$ but
  ``some\_test2'' evaluates to $true$. It executes ``do\_something'' if both
  ``some\_test1'' and ``some\_test2'' evaluate to $false$. 


Give a set of new sound typing rules that we can add to the Cool specification
to accomodate these two new constructs. 



\item The Java programming language includes arrays.  The Java
language specification states that if $s$ is an array of elements of
class $S$, and $t$ is an array of elements of class $T$, then the
assignment $s = t$ is allowed as long as $T$ is a subclass of $S$.
This typing rule for array assignments turns out to be unsound. (Java
works around the fact that this rule is not statically sound by inserting
runtime checks to generate an exception if arrays are used
unsafely. For this question, assume there are no special runtime checks.)

Consider the following Java program, which type checks according
to the preceeding rule:
\begin{verbatim}
class Mammal { String name; }

class Dog extends Mammal { void beginBarking() { ... } }

class Main {
    static public void main(String argv[]) {
        Dog x[] = new Dog[5];
        Mammal y[] = x;

        /* Insert code here */
    }
}
\end{verbatim}
Add code to the \texttt{main} method so that the resulting program is
a valid Java program (i.e., it type checks statically and so it will
compile), but the program could result in an operation being applied
to an inappropriate type when executed.  Include a brief explanation
of how your program exhibits the problem.


\end{enumerate}

\end{document}

\documentclass{article}

\usepackage{../handout}
\usepackage{fullpage}

\newcommand{\tcrule}[3]{\frac{#1}{#2}\eqno\mbox{#3}}

% Handout Numbers

\newcommand{\PAOneNum}{1}
\newcommand{\WAOneNum}{2}
\newcommand{\PATwoNum}{3}
\newcommand{\PAThreeNum}{4}
\newcommand{\WATwoNum}{5}
\newcommand{\WATwoSolNum}{6}
\newcommand{\WAOneSolNum}{7}
\newcommand{\WAThreeNum}{8}
\newcommand{\WAFourNum}{9}
\newcommand{\WAThreeSolNum}{10}
\newcommand{\PAFourNum}{11}
\newcommand{\WAFourSolNum}{12}
\newcommand{\WAFiveNum}{13}
\newcommand{\WAFiveSolNum}{14}
\newcommand{\WASixNum}{15}
\newcommand{\WASixSolNum}{16}
\newcommand{\PAFiveNum}{17}
\newcommand{\WASevenNum}{18}
\newcommand{\WASevenSolNum}{19}
\newcommand{\PAExtraNum}{20}
\newcommand{\WAEightNum}{21}
\newcommand{\WAEightSolNum}{22}
\newcommand{\PASixNum}{23}
\newcommand{\WANineNum}{24}
\newcommand{\WANineSolNum}{25}
\newcommand{\WATenNum}{26}
\newcommand{\WATenSolNum}{27}
\newcommand{\WAElevenNum}{28}
\newcommand{\WAElevenSolNum}{29}


\begin{document}
\handout{\WAElevenNum}{2}{Written Assignment 11 \\ Due May 6, 2004}

This assignment asks you to prepare written answers to questions on
exceptions, higher-order functions, polymorphism, and garbage collection.
Each of the questions has a short
answer.  You may discuss this assignment with other students and work
on the problems together.  However, your write-up should be your own
individual work.  {\em Please write the name of the account you are
using for CS164 {\bf and your section time} on your homework.}
Remember that written assignments are to be turned in either in class
or in the CS164 homework box in 283 Soda by 12:30 PM on the due date.

\bigskip

\begin{enumerate}

\item

Suppose that we have already updated Cool to support
\texttt{try}/\texttt{catch}, \texttt{try}/\texttt{finally}, and
\texttt{throw} as in PA6. Now we wish to add a new construct,
\texttt{protect} $e$, to the language.  At run time, \texttt{protect} $e$
simply returns the value of $e$; however, at compile time, the type
checker must verify that no exceptions can be thrown out of $e$.
(Exceptions may be thrown in the process of
evaluating $e$, but they must be caught before $e$ finishes evaluating.)

In order to check \texttt{protect} expressions at compile time, we need to
extend our typing judgment to track not only the type $T$ of an
expression, but also a boolean $E$ indicating whether an exception
can be thrown out of the expression.

$$
O,M,C \vdash e : T, E
$$

For example, the extended rule for $+$ would be:

$$
\tcrule{
\begin{array}{l}
O,M,C \vdash e_1 : Int, E_1 \\
O,M,C \vdash e_2 : Int, E_2
\end{array}}
{O,M,C \vdash e_1 + e_2 : Int , E_1 \vee E_2 }
{[Plus]}
$$

Give the extended type rules for \texttt{protect},
\texttt{try}/\texttt{catch}, \texttt{try}/\texttt{finally}, and
\texttt{throw}.

\medskip

\item

Consider the following Cool code for a linked list of integers:

\begin{verbatim}
    class List {
        value : Int;
        next : List;
        getValue() : Int { value };
        isSorted() : Bool {
            (isVoid next) ||
            (value <= next.getValue() && next.isSorted())
        };
    };
\end{verbatim}

Note that the short-circuit logical operators {\tt ||} and {\tt \&\&} are
not part of the original Cool language, but we have used them here for the
sake of clarity.

\begin{enumerate}

\item The {\tt isSorted} method only checks that the integers are in
ascending order, but the user may want to check that the integers are in
descending order instead.  Use {\em higher-order functions} (as described
in lecture) to write a new {\tt isSorted} method that allows the caller of
your method to specify how to compare two integers.

\item Imagine that the user really wanted to store a list of strings
instead of a list of integers.  Use {\em polymorphism} (as described in
lecture) to update your solution from part (a) so that the user of your
class can store {\em any} type in the list.  The user should still be able
to specify how to compare these values as in part (a).

\end{enumerate}

\pagebreak

\item

% Note, this problem comes directly from the Fall '96 final, which has
% publicly available solutions (though the solutions provided here are
% probably more helpful). If you care about stopping cheating on the
% homework, you might want to change a constant or something.

% Also, a couple of fixes:

% - n is heap size _in words_

Consider the following simple model for estimating the cost of garbage
collection (GC). We divide running time into time spent actually running
the user's program (which is useful work), and time spent in GC (which is
overhead).

\begin{itemize}

\item The number of cycles required for a single GC is equal to the number
of different memory words that GC references. That is, even if a GC
touches a particular word of memory more than once, we only charge 1
cycle.

\item The \emph{survival rate} is the amount of the live data after a GC,
expressed as a fraction of the total heap size. For example, if garbage
collecting a 10MB heap leaves 1MB of live data, then the survival rate is
0.1.

\item A user program allocates 1 word of heap for every 10 cycles of time
the user program is running (i.e., not doing GC).

\end{itemize}

Given a survival rate $s$ and a total heap size $n$ (measured in words),
give formulas for the fraction of total execution time spent in garbage
collection for Stop-and-Copy and for Mark-and-Sweep. Assume the program
runs for a long time---that is, we are interested in steady-state
performance.

\end{enumerate}

\end{document}

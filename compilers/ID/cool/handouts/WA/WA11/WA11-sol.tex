\documentclass{article}

\usepackage{../handout}
\usepackage{fullpage}
\usepackage{amssymb}

\newcommand{\tcrule}[3]{\frac{#1}{#2}\eqno\mbox{#3}}

% Handout Numbers

\newcommand{\PAOneNum}{1}
\newcommand{\WAOneNum}{2}
\newcommand{\PATwoNum}{3}
\newcommand{\PAThreeNum}{4}
\newcommand{\WATwoNum}{5}
\newcommand{\WATwoSolNum}{6}
\newcommand{\WAOneSolNum}{7}
\newcommand{\WAThreeNum}{8}
\newcommand{\WAFourNum}{9}
\newcommand{\WAThreeSolNum}{10}
\newcommand{\PAFourNum}{11}
\newcommand{\WAFourSolNum}{12}
\newcommand{\WAFiveNum}{13}
\newcommand{\WAFiveSolNum}{14}
\newcommand{\WASixNum}{15}
\newcommand{\WASixSolNum}{16}
\newcommand{\PAFiveNum}{17}
\newcommand{\WASevenNum}{18}
\newcommand{\WASevenSolNum}{19}
\newcommand{\PAExtraNum}{20}
\newcommand{\WAEightNum}{21}
\newcommand{\WAEightSolNum}{22}
\newcommand{\PASixNum}{23}
\newcommand{\WANineNum}{24}
\newcommand{\WANineSolNum}{25}
\newcommand{\WATenNum}{26}
\newcommand{\WATenSolNum}{27}
\newcommand{\WAElevenNum}{28}
\newcommand{\WAElevenSolNum}{29}


\begin{document}
\handout{\WAElevenSolNum}{3}{Written Assignment 11 \\ Solutions}

\begin{enumerate}

\item

The simplest rule is the one for \texttt{protect}:

$$
\tcrule{
\begin{array}{l}
O,M,C \vdash e : T_1, false
\end{array}}
{O,M,C \vdash \mbox{\texttt{protect} } e : T_1, false }
{[Safe]}
$$

That is, $\mbox{\texttt{protect} } e$ has the same type as $e$, but only
type checks if $e$ is known not to throw any exceptions.

Next, the rule for \texttt{throw}:

$$
\tcrule{
\begin{array}{l}
O,M,C \vdash e : T_1, E_1
\end{array}}
{O,M,C \vdash \mbox{\texttt{throw} } e : T_2, true }
{[Throw]}
$$

We know this construct will throw an exception regardless of what $e$
does, so we ignore $E_1$ and return $true$.

Next, the rule for \texttt{try}/\texttt{catch}:

$$
\tcrule{
\begin{array}{l}
O,M,C \vdash e_1 : T_1, E_1 \\
O[T/x],M,C \vdash e_2 : T_2, E_2 \\
\end{array}}
{O,M,C \vdash \mbox{\texttt{try} } e_1 \mbox{ \texttt{catch} } x:T \Rightarrow e_2 : T_1 \sqcup T_2, E_2}
{[Try/Catch]}
$$

Since any exception thrown by $e_1$ will be caught, this construct only
throws an exception if $e_2$ throws an exception.

Finally, the rule for \texttt{try}/\texttt{finally}:

$$
\tcrule{
\begin{array}{l}
O,M,C \vdash e_1 : T_1, E_1 \\
O,M,C \vdash e_2 : T_2, E_2 \\
\end{array}}
{O,M,C \vdash \mbox{\texttt{try} } e_1 \mbox{ \texttt{finally} } e_2 : T_2 \sqcup T_2, E_1 \vee E_2}
{[Try/Finally]}
$$

That is, if either $e_1$ or $e_2$ can throw an exception, then the whole
thing can throw an exception.

\medskip

\item

\begin{enumerate}

\item

We add a \texttt{compare} argument to the \texttt{isSorted} function.  To
simulate a two-argument function, we create a function with type
\texttt{Int -> (Int -> Bool)}.

How do we use this function?  First, note that \texttt{compare(value)} has
type \texttt{Int -> Bool}, and it is a function that compares
\texttt{value} to its argument.  Thus,
\texttt{(compare(value))(next.getValue())} applies this function to the
next value, returning a \texttt{Bool} that indicates the result of the
comparison.

\begin{verbatim}
    class List {
        value : Int;
        next : List;
        getValue() : Int { value };
        isSorted(compare : Int -> (Int -> Bool)) : Bool {
            (isVoid next) ||
            ((compare(value))(next.getValue()) && next.isSorted())
        };
    };
\end{verbatim}

To implement the original comparison operator, we could create the
following function:

\begin{verbatim}
    let lessThan : Int -> (Int -> Bool) <-
        fun (arg1 : Int) : Int -> Bool {
            fun (arg2 : Int) : Bool {
                arg1 <= arg2
            }
        }
\end{verbatim}

By the way, this technique for implementing a two-argument function using
one-argument functions is called {\em currying}.

\item

To make this class polymorphic, we just need to add a type parameter
\texttt{t} and then change \texttt{Int} to \texttt{t} wherever necessary.

\begin{verbatim}
    class List<t> {
        value : t;
        next : List<t>;
        getValue() : t { value };
        isSorted(compare : t -> (t -> Bool)) : Bool {
            (isVoid next) ||
            ((compare(value))(next.getValue()) && next.isSorted())
        };
    };
\end{verbatim}

The \texttt{lessThan} function above can be passed to the
\texttt{isSorted} method of any object of type \texttt{List<Int>}.

\end{enumerate}

\medskip

\item

One good way to solve a problem like this is to break it into sub-problems:

\begin{itemize}
\item When do we need to do GC?
\item How long will GC take?
\item How long will the the program run between GC's?
\end{itemize}

Let's start with Stop-and-Copy:

\begin{itemize}

\item We need to do GC when From space fills up (remember, we divide the
heap in half, into From and To space).

\item Stop-and-Copy only bothers with the words in objects that are still
live (that is, $s \cdot \frac{n}{2}$ words), but we have to copy them
(effectively referencing the copied word as well), so it takes $s \cdot
\frac{n}{2} \cdot 2 = sn$ cycles to do GC.

\item The program will now continue running until it fills up the new From
space. From space can hold $\frac{n}{2}$ words, but $s \cdot \frac{n}{2}$
of those words are filled with objects that were live from the last GC. So
the program can run until it creates $\frac{n}{2} - s \cdot \frac{n}{2}$
words, which takes $10 \cdot (\frac{n}{2} - s \cdot \frac{n}{2}) = 5n -
5sn$ cycles.

\end{itemize}

After this, the machine will do the exact same thing over and over again,
spending $sn$ cycles on GC and $5n - 5sn$ cycles running the program, so
we can use one iteration (GC + running the program) as a model for the
steady state of the program.

One iteration takes $sn + 5n - 5sn$ = $5n - 4sn$ cycles, so the fraction
of total time spent in GC is

$$
\frac{sn}{5n - 4sn} = \frac{s}{5 - 4s}
$$

Solving the problem for Mark-and-Sweep works the same way:

\begin{itemize}

\item Mark-and-Sweep needs to do GC only when the entire heap fills up.

\item Mark-and-Sweep touches all the words in the heap during the sweep
phase, so this takes $n$ cycles.

\item The program will continue running until the heap fills up again.
$sn$ of those words are filled with objects that survived the last GC, so
the program only gets to create $n - sn$ words, which takes $10 \cdot (n -
sn) = 10n - 10sn$ cycles.

\end{itemize}

Thus, the fraction of time spent in GC, using Mark-and-Sweep, is 

$$
\frac{n}{n + 10n - 10sn} = \frac{n}{11n - 10sn} = \frac{1}{11 - 10s}
$$

Note that neither result depends on $n$, the size of the heap.

\end{enumerate}

\end{document}

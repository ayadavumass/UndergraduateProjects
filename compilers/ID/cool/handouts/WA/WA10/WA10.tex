\documentclass{article}

\usepackage{../handout}
\usepackage{fullpage}
\usepackage{graphicx}
\usepackage{amssymb}

% Handout Numbers

\newcommand{\PAOneNum}{1}
\newcommand{\WAOneNum}{2}
\newcommand{\PATwoNum}{3}
\newcommand{\PAThreeNum}{4}
\newcommand{\WATwoNum}{5}
\newcommand{\WATwoSolNum}{6}
\newcommand{\WAOneSolNum}{7}
\newcommand{\WAThreeNum}{8}
\newcommand{\WAFourNum}{9}
\newcommand{\WAThreeSolNum}{10}
\newcommand{\PAFourNum}{11}
\newcommand{\WAFourSolNum}{12}
\newcommand{\WAFiveNum}{13}
\newcommand{\WAFiveSolNum}{14}
\newcommand{\WASixNum}{15}
\newcommand{\WASixSolNum}{16}
\newcommand{\PAFiveNum}{17}
\newcommand{\WASevenNum}{18}
\newcommand{\WASevenSolNum}{19}
\newcommand{\PAExtraNum}{20}
\newcommand{\WAEightNum}{21}
\newcommand{\WAEightSolNum}{22}
\newcommand{\PASixNum}{23}
\newcommand{\WANineNum}{24}
\newcommand{\WANineSolNum}{25}
\newcommand{\WATenNum}{26}
\newcommand{\WATenSolNum}{27}
\newcommand{\WAElevenNum}{28}
\newcommand{\WAElevenSolNum}{29}


\newtheorem{theorem}{Theorem}
\newcommand{\thmLabel}[1]{{\rm (}{\em #1\/}{\rm )}}

\newcommand{\ntsym}[1]{\hbox{\em #1}}
\newcommand{\tsym}[1]{\hbox{\rm #1}}

\def\ra{\rightarrow}     % grammar "rewrites-as" symbol
\def\ep{\varepsilon}     % epsilon for empty-string


\begin{document}
\handout{\WATenNum}{2}{Written Assignment 10 \\ Due April 29, 2004}

This assignment asks you to prepare written answers to questions on
register allocation and instruction scheduling.  Each of the questions has a short
answer.  You may discuss this assignment with other students and work
on the problems together.  However, your write-up should be your own
individual work.  {\em Please write the name of the account you are
using for CS164 {\bf and your section time} on your homework.}
Remember that written assignments are to be turned in either in class
or in the CS164 homework box in 283 Soda by 12:30 PM on the due date.


\begin{enumerate}
\item Consider the following program:

\begin{verbatim}
L0: e := 0
    b := 1;
    d := 2;
L1: a := b+2
    c := d+5
    e := e + c
    f := a * a
    if f < c goto L3
L2: e := e + f
    goto L4
L3: e := e + 2
L4: d := d + 4
    b := b - 4
    if b != d goto L1
L5:
\end{verbatim}

This program uses six temporaries \texttt{a}-\texttt{f}.  Assume that
our machine has only 4 available registers \texttt{\$r0},
\texttt{\$r1}, \texttt{\$r2}, and \texttt{\$r3} 
and that only
\texttt{e} is live on exit from this program.

\begin{enumerate}
\item Draw the register interference graph.  (Computing the sets of
live variables at every program point may be helpful for this step.)

\item Use the graph coloring heuristics discussed in lecture to assign
  each temporary to a register on a machine that has $4$ registers. 
Rewrite the program replacing temporaries by registers and including whatever
spill code is necessary.  Use the pseudo-instructions \texttt{load x}
and \texttt{store x} to load and spill the value of \texttt{x} from
memory.
\end{enumerate}

\item Suppose we have a machine with three registers, two integer units and
one load/store unit with the following cycle counts (and no
out-of-order execution):
\begin{center}
\begin{tabular}{|c|c|} \hline
Addition & 1 cycle \\ \hline
Multiplication & 2 cycles \\ \hline
Branch & 2 cycles \\ \hline
Load/store & 3 cycles \\ \hline
\end{tabular}
\end{center}
Consider the following basic block:
\begin{verbatim}
a := load 0($fp)
b := load 4($fp)
c := a + b
d := b * 42
e := d + 2
\end{verbatim}
Assume only \texttt{a}, \texttt{c}, and \texttt{e} are live on exit to
this block.  

\begin{enumerate} 
\item Draw the instruction dependence graph for the basic
  block shown above. Label each instruction with
  the priority $\langle L,D \rangle$ as defined in
  lecture. Perform instruction scheduling for the above basic block using
  the heuristics discussed in lecture. Note that the heuristics is
  optimal for this basic block.
\item Given an example of a basic block for which this heuristics does
  not yield the most optimal scheduling. Explain your answer. 
\item Give an example of a control flow graph for which the best
  instruction scheduling for each basic block does not yield the best
  instruction schedule for the entire control flow graph. 
\end{enumerate}

\item In class we discussed how to perform register allocation and
instruction scheduling.  Unfortunately, these two important back-end
components do not always work well together.  This problem asks you to
show how this can happen. Assume that the underlying machine is as described in
Question 2.
\begin{enumerate}

\item If register allocation is done before instruction scheduling, it
may assign multiple temporaries to the same register, which can
introduce ``false'' dependencies between variables that can
interfere with scheduling.  Explain why it is better to perform instruction
scheduling on the basic block in Question 2 before register allocation.


\item On the other hand, it's not always a good idea to do instruction
scheduling first, because instruction scheduling can increase the
lifetimes of temporaries.  Give an example program that fits into
three registers if register allocation is done before instruction
scheduling, but requires more than three registers (and hence a spill
on our machine) if register allocation is done after instruction
scheduling.
\end{enumerate}


\end{enumerate}

\end{document}

\documentclass[11pt]{article}
\usepackage{handout}
%
% Copyright (c) 1995-1996 by Alex Aiken.  All rights reserved.
% Permission is granted to modify and distribute this document for
% for non-commercial purposes, so long as this copyright notice is retained
% in all copies.
%
\input{mymargins}
\input{macros}
% Handout Numbers

\newcommand{\PAOneNum}{1}
\newcommand{\WAOneNum}{2}
\newcommand{\PATwoNum}{3}
\newcommand{\PAThreeNum}{4}
\newcommand{\WATwoNum}{5}
\newcommand{\WATwoSolNum}{6}
\newcommand{\WAOneSolNum}{7}
\newcommand{\WAThreeNum}{8}
\newcommand{\WAFourNum}{9}
\newcommand{\WAThreeSolNum}{10}
\newcommand{\PAFourNum}{11}
\newcommand{\WAFourSolNum}{12}
\newcommand{\WAFiveNum}{13}
\newcommand{\WAFiveSolNum}{14}
\newcommand{\WASixNum}{15}
\newcommand{\WASixSolNum}{16}
\newcommand{\PAFiveNum}{17}
\newcommand{\WASevenNum}{18}
\newcommand{\WASevenSolNum}{19}
\newcommand{\PAExtraNum}{20}
\newcommand{\WAEightNum}{21}
\newcommand{\WAEightSolNum}{22}
\newcommand{\PASixNum}{23}
\newcommand{\WANineNum}{24}
\newcommand{\WANineSolNum}{25}
\newcommand{\WATenNum}{26}
\newcommand{\WATenSolNum}{27}
\newcommand{\WAElevenNum}{28}
\newcommand{\WAElevenSolNum}{29}


% when updating this file for a new semester, remember to:
%   - change the DUE DATE (very important!)
%   - update handout.sty, where it says which semester this is

% you may also want to remove the section about the optional tool
%  at the bottom of the document, it was used in spring 2002

\begin{document}
\handout{\PAOneNum}{3}{Programming Assignment I \\ Due Friday, January 30, 2004 at 11:59pm}

This assignment asks you to write a short Cool program.  The purpose is
to acquaint you with the Cool language and to give you experience with
some of the tools used in the course.  This assignment will {\em not} be
done with a partner; you should turn in your own individual work.  All
future programming assignments will be done in teams of either one or
two.

A machine with only a single stack for storage is a {\em stack machine}.
Consider the following very primitive language for programming  
a stack machine: \\
\begin{center}
\begin{tabular}{r|l}
{\em Command} & {\em Meaning} \\ \hline
{\em int}  & push the integer {\em int}  on the stack \\
+ & push a `+' on the stack \\
s & push an `s' on the stack \\
e & evaluate the top of the stack (see below) \\
d & display contents of the stack \\
x & stop
\end{tabular}
\end{center}

The `d' command simply prints out the contents of the stack, one element
per line, beginning with the top of the stack.  The behavior of the `e'
command depends on the contents of the stack when `e' is issued:
\begin{itemize}
\item If `+' is on the top of the stack, then the `+'
is popped off the stack, the following two integers are popped
and added, and the result is pushed back on the stack.

\item If `s' is on top of the stack, then the `s' is popped and
the following two items are swapped on the stack.

\item If an integer is on top of the stack or the stack is empty, the
stack is left unchanged.
\end{itemize}

The following examples show the effect of the `e' command in various
situations; the top of the stack is on the left:
\[
\begin{array}{lcl}
\m{\em stack before} & & \m{\em stack after} \\
+ \; 1 \; 2 \; 5 \; s \ldots & \;\; & 3 \; 5 \; s \ldots \\
s \; 1 \; + \; + \,\; 99 \ldots & & + \,\; 1\; + \; 99 \\
1 \; + \; 3 \ldots & & 1 \; + \; 3 \ldots \\
\end{array}
\]

You are to implement an interpreter for this language in Cool.  Input to
the program is a series of commands, one command per line.  Your
interpreter should prompt for commands with {\tt >}.  Your program need
not do any error checking: you may assume that all commands are valid
and that the appropriate number and type of arguments are on the stack
for evaluation. You may also assume that the input integers are
unsigned.  Your interpreter should exit gracefully; do not call {\tt abort()}
after receiving an {\tt x}. % 'cause abort() confuses the autograder.

You are free to implement this program in any style you choose.
However, in preparation for building a Cool compiler, we recommend that
you try to develop an object-oriented solution.  One approach is to
define a class {\tt StackCommand} with a number of generic operations, and then
to define subclasses of {\tt StackCommand}, one for each kind of command in the
language.  These subclasses define operations specific to each command,
such as how to evaluate that command, display that command, etc.  If you
wish, you may use the classes defined in {\tt atoi.cl} in the {\tt \~{
}cs164/examples} directory to perform string to integer conversion.
If you find any other code in {\tt \~{}cs164/examples} that you think
would be useful, you are free to use it as well.

We wrote a solution in approximately 200 lines of Cool source code.
This information is provided to you as a rough measure of the amount of
work involved in the assignment---your solution may be either
substantially shorter or longer.

\section*{Sample session} 
The following is a sample compile and run of our solution.
\begin{verbatim}
%coolc stack.cl atoi.cl
%spim -file stack.s
SPIM Version 5.6 of January 18, 1995
Copyright 1990-1994 by James R. Larus (larus@cs.wisc.edu).
All Rights Reserved.
See the file README a full copyright notice.
Loaded: /home/ff/cs164/lib/trap.handler
>1
>+
>2
>s
>d
s
2
+
1
>e
>e
>d
3
>x
COOL program successfully executed
\end{verbatim}

\section*{Getting and turning in the assignment} 
Create a working directory called PA1 and {\tt cd} into it.  From there, type
\begin{verbatim}
% gmake -f ~cs164/assignments/PA1/Makefile
\end{verbatim}

This command creates several files you will need in the directory.
Follow the directions in the README file.  

\medskip
%%We are currently testing the submission script.  Instructions for
%%submitting your assignment will be posted to the web page.
To turn the assignment in:

\begin{enumerate}
\item Make sure your code is in stack.cl and that it compiles and works :-).
  A copy of \t{Makefile} and \t{atoi.cl} will be present when we test your
  submission, so all we need from you is \t{stack.cl} and \t{README}.
\item Answer the 3 required questions in the \t{README file}, and
include any other relevant comments here.
\item Make sure everything is in a directory called \t{PA1}.
\item Run the command \t{submit PA1} (case is important) from your PA1
  directory.
\end{enumerate}

Please note: Your stack machine will be tested by comparing its output
to that of our reference implementation.  Therefore, your stack
machine should not produce any output aside from whitespace (which our
testing harness will ignore), `{\tt >}' prompts, and the output of a
`d' command.  Prior to submitting, please remove any output commands
that you used for debugging.

\section*{An optional tool that may be of use to you}


Prof.~Susan L.~Graham's Harmonia project provides Harmonia-mode, an
XEmacs extension that can assist you when writing Cool programs.
Harmonia-mode offers a number of useful features, including syntax
highlighting and proper indentation.  Harmonia-mode will also
highlight syntactic and semantic errors as you type.

Note: Harmonia-mode is research technology and may crash on you;
however, it has been engineered to recover from any type of crash,
preserving your work.  {\em Its use in this class is completely
optional}.  We encourage you to try it out at
{\tt http://harmonia.cs.berkeley.edu/harmonia/cs164}.  If you like
it (or don't), tell us in your README.  Also, please feel free to
email the Harmonia group at {\tt harmonia-feedback@sequoia.cs.berkeley.edu}.

%%Cool is a somewhat unique language.  Until recently, there have
%%been no editing aids (e.g. an Emacs mode) to help you write your
%%programs.  However, this situation has been remedied by some research
%%technology from Prof. Susan L. Graham's Harmonia project.  It's an
%%extension to XEmacs called Harmonia-mode.  Harmonia-mode will analyze
%%your Cool program as you're editing it and give you syntax
%%highlighting, proper indentation, and most importantly, will highlight
%%syntax and semantics errors as you type.
%%
%%Note: Harmonia-mode is research technology, and may crash on you
%%(though, it has been engineered to recover from any type of crash and
%%preserve your work).  {\em Its use in this class is completely 
%%optional}.
%%We encourage those souls who like to live on the bleeding-edge to try
%%it out at http://www.cs.berkeley.edu/\~{}harmonia/harmonia/cs164.  If you like it
%%(or don't) tell us in your README.  Feel free to email the Harmonia
%%group as well, at harmonia-feedback@sequoia.cs.berkeley.edu.

If you used Harmonia-mode for all or part of your project,
please take a minute to answer the following questions in your README:

\begin{enumerate}
\item Rate Harmonia-mode's usefulness on a scale of 0 to 6, where 0 
means
that it kept you from doing any work, 3 means that it neither helped
nor hindered you, and 6 means that you'll never again dare to edit a
Cool program without it.
\item What aspects of Harmonia-mode particularly helped you?
\item What aspects were particularly detrimental?
\item Is there anything else about Harmonia-mode that you would like
to mention?
\end{enumerate}



\section*{Extra Credit.} There is a chance that you will discover a bug in
our Cool compiler.  We will award extra credit for legitimate bug
reports; to get credit, send a bug report to {\tt cs143-aut0506-staff@lists.stanford.edu}.  Your
report must include all of the needed Cool source and a transcript of a
terminal session showing how to reproduce the bug (use the {\tt script}
command).  There are a number of ways the compiler can potentially fail:
the compiler may dump core, the generated code may be incorrect, the
compiler may refuse to accept a legal program, it may accept an illegal
program, etc.  {\em Please be sure you have found a bug before
submitting a report!} The course staff are the final arbiters of what is
a ``bug'' and what is a ``feature''.  Credit usually will be awarded
only to the first person to report a bug.


\end{document}
